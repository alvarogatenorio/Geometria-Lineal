\chapter{Álgebra Lineal}
Este apéndice está especialmente pensado para los alumnos de los dobles grados, que, a fecha de escribir este texto, cursan la asignatura de geometría lineal y la de álgebra lineal con un año de separación.

Este hecho añade a la presente materia un plus de dificultad, pues hace echar mano constantemente de la bibliografía de primer curso, que, en muchas ocasiones, no es sufiente, por ejemplo en el estudio de la \ti{dualidad}.

Algo que merece la pena recalcar es que aquí únicamente se incluyen los resultados más elementales acerca de dualidad, ya que sabemos por experiencia que los resultados más profundos se omiten en un primer curso de álgebra lineal (a pesar de ser harto necesarios aquí). Es por esta razón, no hacer visitar un apéndice al lector sin necesidad, que estos conceptos gozan de sección propia en el texto ordinario.

El objetivo de este anexo no es otro que recopilar los conceptos y resultados que consideramos totalmente imprescindibles para seguir el texto, no obstante, no pretende ser, ni mucho menos, tan completo o rico en ejemplos como otros títulos específicos de álgebra lineal que se recomiendan en la bibliografía.
\section{Coordenadas en Espacios Vectoriales}
El objetivo de esta sección es servir como pequeño área de repaso a la hora de entrar en conceptos íntimamente ligados con los cambios de base en espacios vectoriales. Un ejemplo claro de esto son los cambios de referencia proyectiva.

Sea $E$ un $\K$--espacio vectorial de dimensión finita $n$.

Asimismo, consideraremos la base $\mc{B}:=\{b_1,\dots,b_n\}$ de $E$.

\begin{prop}[Escritura Única de un Vector]
	\label{A1_prop_escrituraUnicaVector}
	Dado un vector $u\in E$, este tiene una escritura \tb{única} como combinación lineal de los vectores de la base $\mc{B}$.
\end{prop}
\begin{proof}
	La existencia de esta escritura es evidente, por ser $\mc{B}$ una base de $E$, y, por tanto, un sistema de generadores. En consecuencia, lo único que hay que probar es la unicidad de dicha combinación lineal. En efecto, supongamos que hubiera dos:
	\[u=\alpha_1b_1+\dots+\alpha_nb_n=\beta_1b_1+\dots+\beta_nb_n
	\]
	Pasando todo al segundo miembro y sacando factor común obtenemos:
	\[(\alpha_1-\beta_1)b_1+\dots+(\alpha_n-\beta_n)b_n=0\]
	Como los vectores de la base son linealmente independientes, se tiene que todos los coeficientes deben ser nulos. Es decir:
	\[\begin{array}{lr}
	\alpha_i-\beta_i=0 & \forall i\in\{1,\dots,n\}
	\end{array}\]
	De donde se sigue la necesaria igualdad de ambas escrituras.
\end{proof}
\begin{obs}[Coordenadas de un Vector Respecto de una Base]
	\label{A1_obs_coordenadasVector}
	Es evidente que, \tb{fijada una base}, todo vector queda caracterizado por su escritura como combinación lineal de los vectores de dicha base. Es por este motivo que, dado un vector $u\in E$ cualquiera, emplearemos la siguiente notación:
	\[
		u = \alpha_1b_1+\dots+\alpha_nb_n\equals{not.}(\alpha_1,\dots,\alpha_n)_{\mc{B}}
	\]
	A la tupla de escalares $(\alpha_1,\dots,\alpha_n)$ la denominaremos \ti{coordenadas de $u$ respecto de la base $\mc{B}$}.
\end{obs}
Por supuesto, si decidimos tomar otra base $\mc{B}'$, las coordenadas de los vectores respecto de la base $\mc{B}'$ serán, en general, distintas a las coordenadas respecto de $\mc{B}$.

Un problema interesante, y que resolveremos en \ref{A1_cambioBase}, consiste en encontrar una relación o ligadura entre ambas coordenadas.

\begin{obs}[Coordenadas del $i$--ésimo Vector de la Base]
	\label{A1_obs_coordenadasVectorBase}
	Dado el vector $b_i$, es interesante notar que sus coordenadas respecto de la base $\mc{B}$, de la que, recordemos, es el $i$--ésimo vector, son:
	\[
	b_i=(0,\dots,\overbrace{1}^{i},\dots,0)_{\mc{B}}
	\]
	La comprobación es inmediata y se deja al lector.
\end{obs}
\subsection{Matriz de Cambio de Base}
\label{A1_cambioBase}
Sean $\mc{B}:=\{e_1,\dots,e_n\}$ y $\mc{B}':=\{e_1',\dots,e_n'\}$ dos bases de un espacio vectorial $E$.
En estas condiciones, dado un vector cualquiera $u\in E$, podemos escribirlo de dos maneras distintas:
\begin{gather}
	u=\alpha_1e_1+\dots+\alpha_ne_n\\
	\label{A1_eq_escritura2}
	u=\beta_1e_1'+\dots+\beta_ne_n'
\end{gather}
Escribiendo cada vector de $\mc{B}'$ como combinación lineal de los vectores de $\mc{B}$, es decir, en coordenadas de $\mc{B}$ obtenemos (los exponentes son símplemente superíndices):
\begin{equation}
	e_i'=\gamma_1^ie_1+\dots+\gamma_n^ie_n
\end{equation}
Uniendo las ecuaciones se tiene:
\begin{multline}
	u=\beta_1(\gamma_1^1e_1+\dots+\gamma_n^1e_n)+\dots+\beta_n(\gamma_1^ne_1+\dots+\gamma_n^ne_n)=\\
	=(\beta_1\gamma_1^1e_1+\dots+\beta_1\gamma_n^1e_n)+\dots+(\beta_n\gamma_1^ne_1+\dots+\beta_n\gamma_n^ne_n)=\\
	=(\beta_1\gamma_1^1+\dots+\beta_n\gamma_1^n)e_1+\dots+(\beta_1\gamma_n^1+\dots+\beta_n\gamma_n^n)e_n
\end{multline}
La traducción de esto a términos de coordenadas nos arroja:
\begin{equation}
	u=(\alpha_1,\dots,\alpha_n)_{\mc{B}}=([\beta_1\gamma_1^1+\dots+\beta_n\gamma_1^n],\dots,[\beta_1\gamma_n^1+\dots+\beta_n\gamma_n^n])_{\mc{B}}
\end{equation}
Esto, por comodidad, lo interpretaremos como producto de matrices (compruébese):
\begin{equation}
	\begin{pmatrix}
	\alpha_1\\
	\vdots\\
	\alpha_n
	\end{pmatrix}=
	\begin{pmatrix}
	\gamma_1^1 & \cdots & \gamma_1^n\\
	\vdots & \ddots & \vdots\\
	\gamma_n^1 & \cdots & \gamma_n^n
	\end{pmatrix}
	\begin{pmatrix}
	\beta_1\\
	\vdots\\
	\beta_n
	\end{pmatrix}
\end{equation}
Usando una notación más compacta:
\begin{equation}
	X_{\mc{B}}=C_{\mc{B}\mc{B}'}X_{\mc{B}'}
\end{equation}
Obsérvese que la matriz $P$ es \tb{cuadrada} e \tb{invertible}, por ser la matriz formada al poner por columnas los vectores de la base $\mc{B}'$ respecto de la base $\mc{B}$.

Por esta razón, podemos despejar $X_{\mc{B}'}$, obteniendo la relación inicialmente buscada:
\begin{equation}
	X_{\mc{B}'}=C_{\mc{B}\mc{B}'}^{-1}X_{\mc{B}}
\end{equation}
A la matriz $C_{\mc{B}\mc{B}'}$ se la denomina \ti{matriz de cambio de base de $\mc{B}$ a $\mc{B}'$}. Es intersante comprobar que su inversa es la matriz de cambio entre las mismas bases en sentido contrario.

Para cerrar la sección diremos, como curiosidad, que toda matriz invertible constituye una matriz de cambio entre ciertas bases.

\section{Ecuaciones de Subespacios}
El objetivo de esta sección será caracterizar un subespacio vectorial por el conjunto de soluciones de una ecuación o conjunto de ecuaciones. A estas ecuaciones las denominaremos \ti{ecuaciones cartesianas}.
\section{Aplicaciones Lineales}
\section{Dualidad}