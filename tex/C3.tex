\chapter{No sé que nombre tendrá}
\label{C3}

Nuestra tarea aquí es tratar de, dado un subespacio proyectivo, por ejemplo una recta o un plano, dar una referencia proyectiva de ese subespacio mediante la cual hacer una descripción explícita de sus elementos. Comenzaremos estudiando el caso más sencillo, las rectas proyectivas.

\section{Rectas Proyectivas}
\begin{defi}[Recta en $\proy(E)$]
	\label{C1_def_rectaProyectiva}
	Se define \ti{recta proyectiva} que pasa por los puntos proyetivos $P$ y $Q$ como la variedad engendrada por dichos puntos. A dicha recta se la denomina \ti{recta} $PQ$.
\end{defi}
\subsection{Ecuaciones paramétricas}

Sean $P=\class{u}$ y $Q=\class{v}$ dos puntos proyectivos, vamos a describir los elementos de la recta $PQ$, que no es otra cosa que $\engen{P,Q}$.
	
Para describir los elementos de esta variedad (o de cualquiera) deberemos dar una referencia en función de la cual \ti{coordenar} todos los puntos de la misma.
	
Como $P$ y $Q$ son dos puntos proyectivos distintos, los vectores $u,v$ son linealmente independientes, formando una base de la variedad lineal $\lengen{u,v}$.
	
Para construir una referencia bastaría tomar los puntos $P,Q$ y añadirle como punto unidad un tercer punto cuyo representante pueda ser escrito como combinación lineal de $u$ y $v$ con todos los coeficientes no nulos, por ejemplo $\class{u+v}$.
	
De esta forma tenemos la referencia:
\[\mf{R}=\{P,Q;\class{u+v}\}\]
Por el método de construcción de bases asociadas tenemos que la base asociada a esta referencia es $\mc{B}=\{u,v\}$. Como sabemos, todo punto $p\in\engen{P,Q}$ es un rayo representado por un vector de $\lengen{u,v}$. Es decir, un vector $w=\alpha u+\beta v$ con alguno de los coeficientes no nulo.
	
Esto quiere decir que todo punto de la recta $PQ$ es un rayo de la forma: \[\class{\alpha u+\beta v}=(\alpha:\beta)\]
Sin embargo, podemos reducir esto aún un poco más, cambiemos el representante del rayo dividiendo todo por $\beta$.
\[\class{\frac{\alpha}{\beta}u+v}:\stackrel{\textrm{not.}}{=}\class{\theta u+v}\]
De esta forma la recta ya no queda descrita por dos coordenadas homogéneas $\alpha$ y $\beta$ como antes, sino por una única coordenada $\theta$ a la que llamaremos \ti{no homogénea}.
	
Sin embargo, hemos de tener cuidado, pues, como más de uno ya se habrá dado cuenta, es posible que en algunos casos $\beta$ se anule, por ende, $\theta$ no estaría definida. Como este caso se corresponde con un único punto, y este es el punto $P$, diremos que una recta queda descrita por lo siguiente:
\begin{equation}
	\label{C3_eq_descripcionRecta}
	PQ:\{\class{\theta u+v}\tq \theta\in\K\}\cup\{P\}
\end{equation}
De esta forma, cuando $\beta=0$ podemos decir que $\theta=\infty$, y así $\theta\in\K\cup \{\infty\}$, que podemos identificar con $\proy^1$. Describiremos pues la recta como
\begin{equation}
	\label{C3_eq_descripcionRectaP1}
	PQ:\{\class{\theta u+v}\tq \theta\in\proy^1\}
\end{equation}
donde se entiende que si $\theta=\infty$ nos estamos refiriendo al punto $P$. 

Dados los vectores $u=(u_0,u_1,\cdots,u_n)$ y $v=(v_0,v_1,\cdots,v_n)$ si los sustituimos en la ecuación~\eqref{C3_eq_descripcionRectaP1} obtenemos
\begin{equation*}
	\label{C_3_eq_parametrica_recta}
	\begin{split}
		PQ:&\{\class{\theta (u_0,u_1,\cdots,u_n)+(v_0,v_1,\cdots,v_n)}\tq \theta\in\proy^1\}=\\
		&=\{\class{(\theta u_0+v_0,\theta u_1+v_1,\cdots,\theta u_n+v_n)}\tq \theta\in\proy^1\}
	\end{split}
\end{equation*}
que se puede escribir a su vez como 
\begin{equation}
	PQ:\{(\theta u_0+v_0:\theta u_1+v_1:\cdots:\theta u_n+v_n)\tq \theta\in\proy^1\}
\end{equation}
denominada \tb{ecuación paramétrica de la recta}. Así, la recta proyectiva está formada por todos aquellos puntos proyectivos que cumplan dicha ecuación, es decir, los que se obtienen al ir variando el valor de $\theta$.
\begin{exa}[Parametrización de una Recta Concreta]
	\label{C3_exa_rectaConcreta}
	Dados los puntos $P=(1:2:-1)$ y $Q=(0:1:3)$ se nos pide parametrizar la recta $PQ$. Siguiendo los pasos expuestos en este apartado, la ecuación paramétrica de la recta $PQ$ queda:
	\[
	PQ:\{(\theta:2\theta+1:-\theta+3)\tq\theta\in\proy^1\}
	\]
	donde, cuando $\theta=\infty$, nos referimos al punto $P=(1:2:-1)$.
	
	Imaginemos que ahora queremos hacernos una idea de donde se encuentra esa recta en $\R^3$, donde están los rayos que pertenecen a ella. Para ello,
\end{exa}
