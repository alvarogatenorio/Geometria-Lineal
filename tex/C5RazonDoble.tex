\chapter{Razón Doble}
El objetivo de este capítulo es ver que, dadas dos rectas proyectivas $\proy(E)$ y $\proy(E')$, con $\dim(E)=\dim(E')=2$,  existe una única homografía que transforma cuatro puntos distintos cualesquiera de $\proy(E)$ en otros cuatro puntos distintos de $\proy(E')$. Para ello iremos construyendo una serie de definiciones y proposiciones a partir de ejemplos y observaciones.

\section{Definición}
Empecemos tratando un caso más sencillo, tres puntos. Es fácil demostrar haciendo uso del álgebra lineal, como haremos a continuación, que, dadas dos rectas proyectivas, existe una única homografía que transforma tres puntos distintos cualesquiera en otros tres puntos distintos.

\begin{prop}
	Dadas dos rectas proyectivas, $\proy(E)$ y $\proy(E')$, y dadas dos ternas diferentes siempre existe una única homografía que transforma la una en la otra.
\end{prop}