\chapter{Clasificación de homografías}
Realizaremos la clasificación de las homografías a partir de sus variedades invariantes. Comenzaremos con las homografías de $\proy^1$, clasificándolas según sus puntos fijos. Relacionaremos esto con las formas canónicas de Jordan y haremos hincapié en algunas homografías importantes, como son las involuciones.

\section{Clasificación de homografías de $\proy^1$. Puntos fijos}
En esta sección haremos una primera aproximación intuitiva a los puntos fijos de una homografía de $\proy^1$, para después enlazarlo con las formas canónicas de Jordan.

Recordemos que podíamos expresar una homografía de $\proy^1$ como
\begin{equation}
	\label{C6_transformada_moebius}
	\theta'=\frac{a\theta+b}{c\theta +d}
\end{equation}
donde $ad-bc\not=0$. Podemos remover un poco las cosas en la ecuación anterior
\begin{equation}
	\theta'(c\theta+d)=a\theta+b\sii c\theta\theta'-a\theta+d\theta'-b=0
\end{equation}
y obtenemos que es equivalente a 
\begin{equation}
	\label{C6_eq_homografia_segundo_orden}
	\alpha\theta\theta'+\beta\theta+\gamma\theta'+\delta=0
\end{equation}
donde $\alpha=c, \ \beta=-a, \ \gamma=d, \ \delta=-b$. 

Que una homografía, distinta de la identidad, mantenga invariante un punto $(x,y)$ implica que $(x',y')=(x,y)$, y con ello $\theta'=\theta$. Sustituyendo en la ecuación anterior obtenemos una ecuación de segundo grado, y, por tanto, tendrá dos soluciones distintas o una solución doble. Esto quiere decir que, suponiendo que $\K=\C$, las homografías, exceptuando la identidad, tienen dos puntos fijos distintos o un punto fijo doble, dependiendo de la ecuación resultante.
\begin{equation}
	\label{C6_eq_homografia_segundo_orden_punto_fijo}
	\alpha\theta^2+(\beta+\gamma)\theta+\delta=0
\end{equation}
Veamos en cada uno de los casos cual es la expresión de la homografía que tiene esos puntos fijos, basándonos en la ecuación anterior.

\subsection{Dos puntos fijos distintos}
Sea $h$ una homografía de $\proy^1$ con dos puntos fijos $M$ y $N$ distintos. Tratamos de encontrar el valor de los parámetros $\alpha, \beta,\gamma,\delta$ para poder describir las homografías de este tipo. Como sabemos que $M$ y $N$ son puntos fijos, sus coordenadas no homogéneas deben cumplir la ecuación~\eqref{C6_eq_homografia_segundo_orden_punto_fijo}. Por tanto una forma de hallar estos parámetro sería sustituir las coordenadas no homogéneas de los puntos fijos en dicha ecuación y ver para que valores de $\alpha, \beta,\gamma,\delta$ se cumple. 

Para facilitar este cálculo tomamos la referencia $\mf{R}=\{M,N;E\}$, donde $E$ es un punto cualquiera distinto de $M$ y $N$. En esta referencia las coordenadas no homogénea de $M$ y $N$ son $\infty$ y $0$, respectivamente. Por tanto la ecuación~\eqref{C6_eq_homografia_segundo_orden_punto_fijo} admite como soluciones $0$ y $\infty$. Sustituyendo estos valores en la ecuación obtenemos que 
\begin{equation*}
	\alpha 0+(\beta+\gamma)0+\delta=0\sii\delta=0
\end{equation*}
Hagamos el cálculo para $\infty$ con más cuidado. Sustituyendo las coordenadas no homogéneas por las coordenadas homogéneas en la ecuación~\eqref{C6_eq_homografia_segundo_orden}, esta queda
\begin{equation}
	\alpha xx'+\beta xy'+\gamma x'y+\delta yy'=0
\end{equation}
Dado que estamos con puntos fijos imponemos que $(x',y')=(x,y)$ y que $\delta=0$, pues lo hemos hallado antes
\begin{equation*}
	\alpha x^2+\beta xy+\gamma xy=0
\end{equation*}
Las coordenadas homogéneas de $M$ respecto a la referencia $\mf{R}$ son $(1:0)$. Sustituyéndolas en la ecuación anterior obtenemos que
\begin{equation*}
\alpha 1+\beta 0+\gamma 0=0\sii \alpha=0
\end{equation*}
Por tanto, en una homografía de $\proy^1$ con dos puntos fijos se tiene que $\alpha=c=0$ y $\delta=-b=0$. Entonces, la \ti{ecuación canónica} de la homografía respecto a la referencia $\mf{R}$ tomada es
\begin{equation}
	\theta'=\frac{a\theta+b}{c\theta +d}=\frac{a}{d}\theta=k\theta
\end{equation}
donde la constante $k$ se llama \ti{módulo} de la homografía. Notemos que dada cualquier homografía que tenga dos puntos fijos distintos, tomando la referencia adecuada, podemos decir que dichos puntos fijos son el cero y el infinito.

\begin{obs}
	El cálculo para el punto $M$, cuya coordenada no homogénea era $\infty$, puede hacerse sin tener tanto cuidado, gracias a que hemos comprobado rigurosamente que funciona.
	\begin{equation*}
		\alpha\theta^2+(\beta+\gamma)\theta+\delta=0\ra\alpha+(\beta+\gamma)\frac{1}{\theta}+\delta\frac{1}{\theta^2}=0
	\end{equation*}
	\begin{equation*}
		\alpha+(\beta+\gamma)\frac{1}{\infty}+\delta\frac{1}{\infty^2}=0\sii \alpha=0
	\end{equation*}
\end{obs}
Es importante observar que, aunque la ecuación $\theta'=k\theta$ se denomina canónica, en realidad no lo es mucho, ya que esta expresión se ha obtenido tomando la referencia $\mf{R}=\{M,N;E\}$. Notemos que si, en vez de tomar como referencia $\mf{R}$, tomamos $\overline{\mf{R}}=\{N,M;E\}$ la ecuación cambia.

En efecto, el cambio de $\mf{R}$ a $\overline{\mf{R}}$ viene dado por la homografía que trasforma $(1:0)$ en $(0:1)$ y $(0:1)$ en $(1:0)$. Sustituyendo estas trasformaciones, en coordenadas no homogéneas, en la ecuación~\eqref{C6_transformada_moebius}, obtenemos que la ecuación de la homografía que transforma $\mf{R}$ en $\overline{\mf{R}}$ es
\begin{equation*}
	\overline{\theta}=\frac{b}{c\theta}=\frac{\lambda}{\theta}
\end{equation*}
Por tanto, la ecuación $\theta'=k\theta$ respecto a la referencia $\overline{\mf{R}}$, que se obtiene despejando en la ecuación anterior $\theta$ en función de $\overline{\theta}$ y sustituyendo, es
\begin{equation*}
	\frac{\lambda}{\overline{\theta}'}=k\frac{\lambda}{\overline{\theta}}\sii \overline{\theta}'=\frac{1}{k}\overline{\theta}
\end{equation*}
por lo que en este caso el módulo de la homografía con puntos fijos $M$ y $N$ sería $1/k$. 

Podríamos preguntarnos entonces porque darle el apellido canónico. Aunque una permutación de los puntos $M$ y $N$ en la referencia nos da un valor del módulo de la homografía distinto, el parámetro $k$ está geométricamente asociado a la homografía, cosa que veremos no ocurrirá con parámetros posteriores. Es decir, solo depende de la homografía y del orden de los puntos $M$ y $N$ en la referencia, que según sea obtendremos $k$ o $1/k$. Observemos que no depende del punto unidad $E$ que hayamos tomado. Por tanto la ecuación
\begin{equation}
	\label{C6_theta_ktheta}
	\theta'=k\theta
\end{equation}
puede considerarse canónica.

El parámetro $k$ puede ser hallado a través de la razón doble, ya que estamos tratando homografías. En efecto, dada una homografía $h$ que deja fijos los puntos $M$ y $N$, y por tanto en la referencia $\mf{R}=\{M,N;E\}$ viene descrita por la ecuación~\eqref{C6_theta_ktheta} para cierto $k$, se tiene que
\begin{equation}
	\{M,N,P,h(P)\}=\{\infty,0,\theta,\theta'=k\theta\}=k
\end{equation}
para un punto $P$ cualquiera independientemente de $E$. Si permutamos los puntos $M$ y $N$, por la simetría de la razón doble se tiene que
\begin{equation}
	\{N,M,P,h(P)\}=\frac{1}{k}
\end{equation}
independientemente de $E$. Esto nos demuestra de nuevo que el módulo depende solo de la homografía y del orden de los puntos fijos.

Hagamos un pequeño inciso que retomaremos más adelante. Si escribimos, en la referencia $\mf{R}=\{M,N,E\}$, la matriz asociada a una homografía $h$ que tiene dos puntos fijos $M$ y $N$ obtenemos
\begin{equation*}
	\left( \begin{array}{cc}
	a&0\\
	0&d
	\end{array}\right) 
	\sim \left( \begin{array}{cc}
	\frac{a}{d}&0\\
	0&1
	\end{array}\right) 
	\sim \left( \begin{array}{cc}
	k&0\\
	0&1
	\end{array}\right)
\end{equation*}
Por tanto, hemos encontrado una referencia en la cual la homografía $h$ es diagonal.

\subsection{Un punto fijo doble}
Veamos ahora que ecuación describe a una homografía $h$ de $\proy^1$ que deja fijo un solo punto. Si $M$ es el punto fijo de $h$, entonces la ecuación~\eqref{C6_eq_homografia_segundo_orden_punto_fijo} debe admitir como solución la coordenada no homogénea de $M$. Por tanto, procederemos como en el caso anterior. Tomaremos una referencia adecuada y sustituiremos la coordenada no homogénea de $M$, respecto a dicha referencia, en la ecuación~\eqref{C6_eq_homografia_segundo_orden_punto_fijo} para hallar los parámetros $\alpha, \beta,\gamma,\delta$.

Dado que solo hay un punto fijo, tomamos como referencia $\mf{R}=\{M,X;E\}$, donde $X$ y $E$ son puntos diferentes y distintos de $M$. En esta referencia la coordenada no homogénea de $M$ es $\infty$. Por tanto, para cualquier homografía con un punto fijo doble, tomando la referencia adecuada, se puede decir que dicha homografía deja fijo el infinito. Dado que ya justificamos anteriormente los cálculo con $\infty$ esta vez lo haremos sin tener cuidado. Recordemos que podíamos escribir
\begin{equation*}
	\alpha\theta^2+(\beta+\gamma)\theta+\delta=0\ra\alpha+(\beta+\gamma)\frac{1}{\theta}+\delta\frac{1}{\theta^2}=0
\end{equation*}
\begin{equation*}
	\alpha+(\beta+\gamma)\frac{1}{\infty}+\delta\frac{1}{\infty^2}=0\sii \alpha=0
\end{equation*}
Por tanto la ecuación que describe los puntos fijos de la homografía pasa a ser
\begin{equation*}
	(\beta+\gamma)\theta+\delta=0
\end{equation*}
Dado que el punto $M$ es solución doble de la ecuación de segundo grado~\eqref{C6_eq_homografia_segundo_orden_punto_fijo}, si volvemos a sustituir su coordenada no homogénea en la ecuación anterior, esta debe cumplirse. Por tanto
\begin{equation*}
	(\beta+\gamma)\theta+\delta=0\ra(\beta+\gamma)+\delta\frac{1}{\theta}=0
\end{equation*}
\begin{equation*}
	(\beta+\gamma)+\delta\frac{1}{\infty}=0\sii \beta=-\gamma
\end{equation*}
Por tanto, en una homografía de $\proy^1$ con un punto fijo doble se tiene que $\alpha=c=0$ y $-a=\beta=-\gamma=-d$. Entonces, la ecuación ``canónica" de la homografía, respecto a la referencia $\mf{R}$ tomada, es
\begin{equation}
\theta'=\frac{a\theta+b}{c\theta +d}=\frac{a\theta+b}{a}=\theta+\mu
\end{equation}
A este tipo de homografías de las denomina \ti{elaciones} y se dice que dejan fijo el infinito, pues, como hemos visto antes, tomando la referencia adecuada el punto fijo $M$ pasa a ser $\infty$.

Notemos que esta vez el apellido canónico no es merecido, al contrario que en el caso anterior, pues el parámetro $\mu$ no está geométricamente asociado a la homografía. 

Veámoslo. Partiendo de la referencia $\mf{R}=\{M,X;E\}$, realizamos un cambio de coordenadas que lleve el infinito al infinito, por ejemplo aplicando una homografía con ecuación
\begin{equation*}
	\overline{\theta}=\frac{1}{k}\theta
\end{equation*}
es decir, una homografía $h$ que en esa referencia deje fijos el cero y el infinito. Por tanto, estamos pasando de la referencia $\mf{R}$ a la referencia $\overline{\mf{R}}=\{M,X;h(E)\}$. 

La ecuación $\theta'=\theta+\mu$ respecto a la referencia $\overline{\mf{R}}$, que se obtiene despejando en la ecuación anterior $\theta$ en función de $\overline{\theta}$ y sustituyendo, es
\begin{equation*}
	k\overline{\theta}'=k\overline{\theta}+\mu\sii \overline{\theta}'=\overline{\theta}+\frac{\mu}{k}
\end{equation*}
Si $k=\mu$ entonces la ecuación de una homografía cuyo punto fijo es $M$ respecto a la referencia $\overline{\mf{R}}$ es
\begin{equation*}
	\overline{\theta}'=\overline{\theta}+1
\end{equation*}
Como se puede observar, el parámetro $\mu$ no está asociado a la homografía, pues lo hemos eliminado de la ecuación que la define. Por tanto, la ecuación $\theta'=\theta+\mu$ no se puede considerar canónica. Lo máximo a lo que podemos aspirar es a expresar la ecuación de una elación como $\theta'=\theta+1$, tomando la referencia adecuada.

De nuevo, hagamos un inciso que retomaremos más adelante. Dada una homografía $h$ que tiene un punto fijo doble $M$, si escribimos, en la referencia $\mf{R}$ en la cual la ecuación que la define es $\theta'=\theta+1$ (y por tanto $b=a$), su matriz asociada obtenemos
\begin{equation*}
\left( \begin{array}{cc}
a&a\\
0&a
\end{array}\right) 
\sim \left( \begin{array}{cc}
1&1\\
0&1
\end{array}\right) 
\end{equation*}
Por tanto, hemos encontrado una referencia en la cual la matriz de la homografía $h$ se encuentra en su forma canónica de Jordan, salvo múltiplo.

\section{Formas canónicas de Jordan y puntos fijos}
Hemos anticipado que existe cierta relación entre los puntos fijos de una homografía y la forma canónica de Jordan de su matriz asociada. En este apartado explicaremos cual es esa relación.\\

Sea $h$ una homografía distinta de la identidad. Si un punto $[v]$ es punto fijo de la homografía, entonces 
\begin{equation}
	\label{C6_eq_puntofijo_autovector}
	[\widehat{h}(v)]=h([v])=[v]\sii \widehat{h}(v)=\rho v
\end{equation}
para cierto $\rho$, de lo cual se deduce que $v$ es autovector de la aplicación lineal $\widehat{h}$ con autovalor asociado $\rho$. Por tanto un punto $x$ es punto fijo de la homografía si y solo si es un autovector de la aplicación lineal $\widehat{h}$, asociado a un autovalor no nulo.

Antes de continuar hagamos un pequeño recordatorio.

\begin{obs}
	Dada un endomorfismo $\widehat{h}$ de $E$ ($\dim(E)=n$) y sean $\lambda_i$, con $i=1,\cdots,r$, sus distintos autovalores, se llama subespacio propio asociado a $\lambda_i$ al subesapcio vectorial $V_{\lambda_i}$ formado por todos los autovectores asociados a dicho autovalor:
	\begin{equation}
		V_{\lambda_i}:=\{u\in E\tq \widehat{h}(u)=\lambda_i u\}
	\end{equation}
	Se verifica que 
	\begin{equation}
		V_{\lambda_i}= Ker(\widehat{h}-\lambda_i I) \ ; \quad \dim(V_{\lambda_i})=n-rg(A-\lambda_i I)
	\end{equation}
	donde $A$ es la matriz asociada a $\widehat{h}$. 
	
	La dimensión de $V_{\lambda_i}$, denotada por $d_i$ se denomina multiplicidad geométrica. Por otro lado, la multiplicidad de $\lambda_i$ como raíz del polinomio característico se denomina multiplicidad algebraica, y se denota por $\alpha_i$. Si $\alpha_i+\cdots+\alpha_r=n$ y $d_i=\alpha_i$ para cada $i=1,\cdots,r$, entonces $\widehat{h}$ es diagonalizable. En el caso en el que solo se cumpla que $\alpha_i+\cdots+\alpha_r=n$ el endomorfismo será ``Jordanizable".
\end{obs}

Sabemos pues que $u$ es autovector de $\widehat{h}$ asociado a un autovalor $\lambda$ no nulo si y solo si $[u]$ es punto fijo. Por tanto, como $V_{\lambda}$ está formado por todos los autovectores asociados a $\lambda$, se tiene que $\proy(V_\lambda)$ es una variedad proyectiva formada por punto fijos de $h$. Además, el conjunto de las proyecciones de los subespacios propios asociados a los distintos autovalores de $\widehat{h}$ nos proporcionan todos los puntos fijos de $h$.

Notemos que hasta ahora no hemos utilizado en absoluto que $h$ sea una homografía de $\proy^1$. Con esto queremos resaltar que todo lo dicho es válido para homografías de $\proy^n$. Es decir, los puntos fijos de una homografía de $\proy^n$ vienen dados por $\proy(V_{\lambda_i})$ para los distintos autovalores $\lambda_i$ de la aplicación lineal asociada.

\begin{obs}
	Cualquier homografía de un espacio proyectivo complejo tiene puntos fijos. Sin embargo, hay homografías del espacio proyectivo real que no los tienen. Podemos precisar un poco más, ya que sabemos que todo polinomio con coeficientes en $\R$ de grado impar tiene raíces. Por tanto para que haya homografías sin puntos fijos, el grado del polinomio debe ser par, con lo cual la dimensión del espacio vectorial es par y, por tanto, la dimensión del espacio proyectivo debe ser impar.
\end{obs}

Indaguemos un poco más para obtener las conclusiones del apartado anterior, es decir, que una homografía $h$ de $\proy^1$ distinta de la identidad tiene o dos puntos fijos distintos o uno doble. Para ello trabajemos con matrices.

Sea $A$ la matriz asociada a la homografía $h$, que no es más que la matriz asociada a la aplicación lineal $\widehat{h}$, y $x$ un punto fijo de $h$. Entonces $\rho x=Ax$, con lo cual se obtiene el sistema $(A-\rho I)x=0$, que tiene solución no idénticamente nula si y solo si $\det(A-\rho I)=0$. Este determinante no es otro que el polinomio característico, cuyas raíces, $\rho$, son los autovalores de la aplicación lineal $\widehat{h}$. Dado que nos encontramos en $\proy^1$, la matriz $A$ será una matriz $2\times 2$, y por tanto el polinomio característico será un polinomio de segundo grado. Esto implica que tendrá dos soluciones distintas o una solución doble, siempre que el cuerpo sobre el que estemos trabajando se $\C$. Por tanto la aplicación lineal $\widehat{h}$ tendrá un autovalor o dos distintos. Evaluemos cada caso.

Si $\widehat{h}$ tiene dos autovalores distintos, entonces $\dim(V_{\rho_i})=1$. Por tanto las variedades proyectivas $\proy(V_{\rho_1})$ y $\proy(V_{\rho_2})$ tienen dimensión cero, por lo que son puntos. Esto da lugar a dos puntos fijos distintos en la homografía. Además si escribimos la matriz $A$ en la base formada por estos dos autovectores obtendremos una matriz diagonal, como habíamos visto antes.

Si $\widehat{h}$ tiene un solo autovalor $\rho$, en cuyo caso $\alpha_1=2$, en principio pueden darse dos casos.

El caso en el que $\dim(V_{\rho})=2=d_1$ no es admisible, ya que hemos supuesto que $h$ es distinta de la identidad. Sin embargo, la aplicación $\widehat{h}$ con $\dim(V_{\rho})=2$ tiene como autovectores asociados a $\rho$ todo el espacio vectorial, por lo que todos los puntos de $\proy^1$ serían puntos fijos de $h$.

Si $\dim(V_{\rho})=1=d_1$ entonces $\proy(V_{\rho})$ es un punto. En tal caso, la homografía $h$ tendrá un único punto fijo doble. Además, dado que $\alpha_1=2\not=d_1=1$, la matriz $A$ no es diagonalizable. Sin embargo, si admite forma canónica de Jordan. Por tanto, en cierta base, a la cual pertenece dicho autovector, la matriz $A$ se encuentra en su forma canónica de Jordan, como vimos anteriormente.\\

Con todo ello podemos concluir que una homografía de $\proy^1$, distinta de la identidad, tiene dos puntos fijos distintos o un punto fijo doble. Si escribimos la matriz asociada a la homografía en la base dada por dichos puntos fijos obtendremos, en el primer caso, una matriz diagonal y, en el segundo, su forma canónica de Jordan. 

Notemos a esta misma conclusión habíamos llegado en el apartado anterior.
\begin{obs}
	Sea una homografía $h$ de $\proy^1$ y su aplicación lineal asociada $\widehat{h}$. Un punto $[v]$ de $\proy^1$ pertenece al centro de $h$ si y solo si es un autovector de $\widehat{h}$ con autovalor $0$.
	
	En efecto, que $[v]$ esté en el centro de $h$ equivale a que $v\in ker\widehat{h}$, es decir, que $\widehat{h}(v)=0=0\cdot v$.
\end{obs}

\section{Aplicaciones afines}
En esta sección y en la que sigue trataremos algunos casos particulares, de gran importancia, de homografías de $\proy^1$.

Estudiemos las homografía de $\proy^1$ tales que su matriz asociada es
\begin{equation*}
	A=\left( \begin{array}{cc}
		a&b\\
		0&d
	\end{array}\right)
\end{equation*}
es decir, $c=0$, y con ello , como $d$ no puede ser cero, 
\begin{equation}
	\theta'=\frac{a\theta+b}{d}=\alpha\theta+\beta
\end{equation}
donde $\alpha=a/d$ y $\beta=b/d$. Veamos cuales son sus puntos fijos.

Siguiendo lo visto en el apartado anterior, debemos encontrar los autovalores de $A$ y sus autovectores asociados. Realizando unas sencillas cuentas se llega a que los autovalores son $a$ y $d$. Dado que hay dos autovalores distintos, la homografía $h$ tendrá dos puntos fijos. 

Calculemos dos autovectores, $u_a$ y $u_d$ asociados a estos autovalores. Recordemos que, como $V_{\lambda}= Ker(\widehat{h}-\lambda I)$, si $u$ es un autovector con autovalor $\lambda$, entonces $u\in V_{\lambda}= Ker(\widehat{h}-\lambda I)$, y por tanto $(A-\lambda I)u=0$. Esto nos proporciona una forma de hallar los autovectores. En nuestro caso
\begin{equation*}
	(A-a I)\left( \begin{array}{c}
	x\\
	y
	\end{array}\right)=\left( \begin{array}{cc}
	0&b\\
	0&d-a
	\end{array}\right)
	\left( \begin{array}{c}
	x\\
	y
	\end{array}\right)=\left( \begin{array}{c}
	0\\
	0
	\end{array}\right)\sii y=0\ra u_{a}=(1,0)
\end{equation*}
\begin{equation*}
	(A-d I)\left( \begin{array}{c}
	x\\
	y
	\end{array}\right)=\left( \begin{array}{cc}
	a-d&b\\
	0&0
	\end{array}\right)
	\left( \begin{array}{c}
	x\\
	y
	\end{array}\right)=\left( \begin{array}{c}
	0\\
	0
\end{array}\right)\sii y=\frac{a-d}{b}x\ra u_b=(b,d-a)
\end{equation*}
Por tanto los dos puntos fijos de la homografía $h$ son $(1:0)$ y $(b/d:1-a/d)=(\beta:1-\alpha)$, cuyas coordenadas no homogéneas son $\infty$ y $\frac{\beta}{1-\alpha}$. 

Dado que la ecuación $\theta'=\alpha\theta+\beta$ define una aplicación afín de la recta podemos concluir que las afinidades, homografía con $c=0$, dejan fijo el infinito. 
\begin{obs}
	No se debe confundir esto con las elaciones. Mientras estás últimas tienen un punto fijo doble, que se puede identificar con el infinito en cierta referencia, las homografías con $c=0$ dejan fijo el infinito sea cual sea la referencia en la que nos encontramos. Es decir, si cambiamos de referencia seguirán teniendo como punto fijo el punto $(1:0)$ de dicha referencia.
\end{obs}
Veamos el recíproco, si una homografía de $\proy^1$ deja fijo el infinito, entonces es una aplicación afín. 

Sea una homografía de $\proy^1$ con matriz asociada
\begin{equation*}
	A=\left( \begin{array}{cc}
		a&b\\
		c&d
	\end{array}\right)
\end{equation*}
Si deja fijo el infinito, entonces
\begin{equation*}
	\left( \begin{array}{cc}
	a&b\\
	c&d
	\end{array}\right)
	\left( \begin{array}{c}
	1\\
	0
	\end{array}\right)=\rho\left( \begin{array}{c}
	1\\
	0
	\end{array}\right)\sii a=\rho \ , \ c=0
\end{equation*}
Por tanto, la matriz de la homografía, dado que $d\not=0$, es
\begin{equation*}
	A=\left( \begin{array}{cc}
		a&b\\
		0&d
	\end{array}\right)\sim 
	\left( \begin{array}{cc}
		\alpha&\beta\\
		0&1
	\end{array}\right)
\end{equation*}
Si describimos la homografía que deja fijo el infinito a través de la transformación de moebius obtenemos que 
\begin{equation*}
	\theta'=\alpha\theta+\beta
\end{equation*}
es decir, es una transformación afín de la recta.

Estos dos resultados dan lugar al siguiente lema.
\begin{lem}\label{C6_lem_infinito_p1_sii_afinidad}
	Dada una homografía de $\proy^1$ y $\mf{R}=\{x_0,x_1;E\}$ una referencia arbitraria, entonces $h$ deja fijo el infinito, es decir deja fijo el punto $x_0$, si y solo si $c=0$, o equivalentemente, si y solo si es una afinidad.
\end{lem}

El calculo de los puntos fijos de la homografía de $\proy^1$ dada al principio de esta sección, aquella cuya matriz asociada es
\begin{equation*}
	A=\left( \begin{array}{cc}
		a&b\\
		0&d
	\end{array}\right)
\end{equation*}
y que viene descrita a través de la ecuación
\begin{equation*}
	\theta'=\alpha\theta+\beta
\end{equation*}
podría haberse hecho como se hizo en el primer apartado de este capítulo, cuando aún no conocíamos la relación entre los puntos fijos y los autovectores.

En tal caso, tendríamos que buscar las coordenadas no homogéneas $\theta$ que cumplen la ecuación
\begin{equation*}
	\theta=\alpha\theta-\beta
\end{equation*}
Sin embargo, esta ecuación, dado que es de primer orden, solo tiene una solución, pero sabemos que el $\infty$, es decir el $(1:0)$, es también punto fijo de $h$. Por tanto, debe cumplir también la ecuación. Si hacemos la comprobación con las coordenadas no homogéneas obtenemos que 
\begin{equation*}
	\infty=\alpha\infty-\beta=\infty
\end{equation*}
Sin embargo, personalmente, esta comprobación no me deja nada tranquila. Por tanto, vamos a comprobarlo de nuevo con las coordenadas homogéneas. En tal caso tenemos la ecuación
\begin{equation}
	\label{C6_puntofijo_malcalculo}
	\frac{x'}{y'}=\alpha\frac{x}{y}-\beta
\end{equation}
\begin{equation}
	\label{C6_puntofijo_buencalculo}
	x'y=\alpha xy'+\beta yy'
\end{equation}
Imponiendo que $(x,y)$ es punto fijo obtenemos 
\begin{equation*}
	 xy=\alpha xy+\beta y^2\sii xy-\alpha xy-\beta y^2=0
\end{equation*}
Una simple comprobación nos permite asegurar que el punto $(1:0)$ es punto fijo de $h$
\begin{equation*}
	1\cdot 0-\alpha 1\cdot 0+\beta0=0
\end{equation*}
\begin{obs}
	Si en vez imponer la condición de punto fijo, $(x,y)=(x',y')$, en la ecuación~\eqref{C6_puntofijo_buencalculo} lo hacemos en la ecuación~\eqref{C6_puntofijo_malcalculo} y luego multiplicamos por $y$, el punto $(1:0)$ solo es solución de la ecuación resultante si $\alpha=1$, cuando debería serlo para todo $\alpha$.
	\begin{equation*}
		\frac{x}{y}=\alpha\frac{x}{y}-\beta\sii x=\alpha x+\beta y
	\end{equation*}
	\begin{equation*}
		 1=\alpha 1+\beta 0=\alpha
	\end{equation*}
	Esto nos muestra que trabajar con las coordenadas no homogéneas y pasar después a homogéneas tiene sus limitaciones, y siempre que hagamos cálculos de este estilo debemos tener bien claro qué estamos haciendo.
\end{obs}

\subsection{Aplicaciones afines y homografías de $\proy^n$}
El lema~\ref{C6_lem_infinito_p1_sii_afinidad} se puede generalizar a una homografía de $\proy^n$. A continuación expondremos y demostraremos dicho resultado.

\begin{prop}
	Sea $h$ una homografía de $\proy^n$ y $\mf{R}=\{x_0,\cdots,x_n;E\}$ una referencia arbitraria. El hiperplano $H$ dado por la ecuación implícita $x_n=0$ se denomina plano del infinito. Si $h$ deja invariante el plano del infinito entonces es una aplicación afín.
\end{prop}
\begin{obs}
	Que una homografía deje invariante un hiperplano $H$ no significa que deje invariante cada punto del hiperplano, es decir que los puntos de $H$ sean puntos fijos, sino que la imagen de un punto de $H$ pertenece al hiperplano, pero no tiene porque ser el mismo punto.
\end{obs}
\begin{proof}
	Sea $h:\proy^n\rightarrow \proy^n$ una homografía que deja invariante el plano $H$ dado por la ecuación $x_n=0$. La matriz asociada a dicha aplicación vendrá dada por
	\begin{equation*}
		\left( \begin{array}{ccc}
			a_{00}&\cdots&a_{0n}\\
			a_{10}&\cdots&a_{1n}\\
			\vdots&\ddots&\vdots\\
			a_{n0}&\cdots&a_{nn}
		\end{array}\right) 
	\end{equation*}
	para ciertos $a_{ij}$, con $i,j=0,\cdots,n$. Dado un punto $(x_0:\cdots:x_n)$ su imagen vendrá dada por la clase del vector $(x_0',\cdots,x_n')$ tal que
	\begin{equation*}
		\left( \begin{array}{ccc}
			a_{00}&\cdots&a_{0n}\\
			a_{10}&\cdots&a_{1n}\\
			\vdots&\ddots&\vdots\\
			a_{n0}&\cdots&a_{nn}
		\end{array}\right) 
		\left( \begin{array}{c}
			x_0\\
			x_1\\
			\vdots\\
			x_n
		\end{array}\right) =\rho
		\left( \begin{array}{c}
			x_0'\\
			x_1'\\
			\vdots\\
			x_n'
		\end{array}\right)
	\end{equation*}
	Los vectores pertenecientes a $\widehat{H}$ tienen la forma $(x_0,\cdots,x_{n-1},0)$. Dado que $h$ deja invariante el hiperplano $H$ debe cumplirse que
	\begin{equation*}
		\left( \begin{array}{ccc}
			a_{00}&\cdots&a_{0n}\\
			a_{10}&\cdots&a_{1n}\\
			\vdots&\ddots&\vdots\\
			a_{n0}&\cdots&a_{nn}
		\end{array}\right) 
		\left( \begin{array}{c}
			x_0\\
			\vdots\\
			x_{n-1}\\
			0
		\end{array}\right) =\rho
		\left( \begin{array}{c}
			x_0'\\
			\vdots\\
			x_{n-1}'\\
			0
		\end{array}\right)
	\end{equation*}
	Por tanto, debe cumplirse la ecuación
	\begin{equation*}
		a_{n0}x_0+\cdots+a_{nn-1}x_{n-1}+a_{nn}\cdot0=\rho\cdot0=0
	\end{equation*}
	para todo $(x_0,\cdots,x_{n-1})\not=(0,\cdots,0)$. Esto implica que, necesariamente
	\begin{equation*}
		a_{n0}=0=a_{n1}=\cdots=a_{nn-1}
	\end{equation*}
	con lo que la matriz asociada a $h$ adopta la forma
	\begin{equation*}
		\left( \begin{array}{cccc}
			a_{00}&\cdots&a_{0n-1}&a_{0n}\\
			a_{10}&\cdots&a_{1n-1}&a_{1n}\\
			\vdots&\ddots&\vdots&\vdots\\
			0&\cdots&0&a_{nn}
		\end{array}\right) 
	\end{equation*}
	Dado que $a_{nn}\not=0$, pues debe ser invertible, la matriz se puede expresar, usando bloques, como
	\begin{equation*}
		\left( \begin{array}{ccc|c}
			&&&\\
			&A&&b\\
			&&&\\ \hline
			0&\cdots&0&1
		\end{array}\right) =
		\left( \begin{array}{ccc|c}
			&&&\\
			&A&&b\\
			&&&\\ \hline
			&0^t&&1
		\end{array}\right)
	\end{equation*}
	Veamos cual es la expresión de una homografía de $\proy^n$ con esta matriz asociada. Dado un punto $p=(x_0:\cdots:x_{n-1}:x_n)\not\in H$, es decir, tal que $x_n\not=0$, podemos escribirlo como
	\begin{equation*}
		p=(x_0:\cdots:x_{n-1}:x_n)=\left( \frac{x_0}{x_n}:\cdots:\frac{x_{n-1}}{x_n}:1\right) =(X_0:\cdots:X_{n-1}:1)=
		\left( \begin{array}{c}
			\\
			X\\
			\\
			1
		\end{array}\right)^t
	\end{equation*}
	donde $X_i=\frac{x_i}{x_n}$ para $i=0,\cdots,n-1$. Entonces
	\begin{equation*}
		\left( \begin{array}{ccc|c}
			&&&\\
			&A&&b\\
			&&&\\ \hline
			&0^t&&1
		\end{array}\right)
		\left( \begin{array}{c}
			\\
			X\\
			\\
			1
		\end{array}\right)=
		\left( \begin{array}{c}
			\\
			AX+b\\
			\\
			1
		\end{array}\right)=
		\left( \begin{array}{c}
			\\
			X'\\
			\\
			1
		\end{array}\right)
	\end{equation*}
	Con ello, la homografía $h$ que deja fijo el plano del infinito $H$ viene descrita por
	\begin{equation}
		h:X'=AX+b
	\end{equation}
	siendo por tanto $h$ una aplicación afín.
\end{proof}




\section{Involuciones}
