\chapter{Clasificación de homografías}
Realizaremos la clasificación de las homografías a partir de sus variedades invariantes. Comenzaremos con las homografías de $\proy^1$, clasificándolas según sus puntos fijos. Relacionaremos esto con las formas canónicas de Jordan y haremos hincapié en algunas homografías importantes, como son las involuciones.

\section{Clasificación de homografías de $\proy^1$. Puntos fijos}
En esta sección haremos una primera aproximación intuitiva a los puntos fijos de una homografía de $\proy^1$, para después enlazarlo con las formas canónicas de Jordan.

Recordemos que podíamos expresar una homografía de $\proy^1$ como
\begin{equation}
	\label{C6_transformada_moebius}
	\theta'=\frac{a\theta+b}{c\theta +d}
\end{equation}
donde $ad-bc\not=0$. Podemos remover un poco las cosas en la ecuación anterior
\begin{equation}
	\theta'(c\theta+d)=a\theta+b\sii c\theta\theta'-a\theta+d\theta'-b=0
\end{equation}
y obtenemos que es equivalente a 
\begin{equation}
	\label{C6_eq_homografia_segundo_orden}
	\alpha\theta\theta'+\beta\theta+\gamma\theta'+\delta=0
\end{equation}
donde $\alpha=c, \ \beta=-a, \ \gamma=d, \ \delta=-b$, denominada \ti{ecuación general} o \ti{implícita} de la homografía.

Que una homografía, distinta de la identidad, mantenga invariante un punto $(x,y)$ implica que $(x',y')=(x,y)$ y con ello $\theta'=\theta$. Sustituyendo en la ecuación anterior obtenemos una ecuación de segundo grado, y, por tanto, tendrá dos soluciones distintas o una solución doble. Esto quiere decir que, suponiendo que $\K=\C$, las homografías, exceptuando la identidad, tienen dos puntos fijos distintos o un punto fijo doble, dependiendo de la ecuación resultante.
\begin{equation}
	\label{C6_eq_homografia_segundo_orden_punto_fijo}
	\alpha\theta^2+(\beta+\gamma)\theta+\delta=0
\end{equation}
Veamos en cada uno de los casos cuál es la expresión de la homografía que tiene esos puntos fijos, basándonos en la ecuación anterior.

\subsection{Dos puntos fijos distintos}
Sea $h$ una homografía de $\proy^1$ con dos puntos fijos $M$ y $N$ distintos. Tratamos de encontrar el valor de los parámetros $\alpha, \beta,\gamma,\delta$ para poder describir las homografías de este tipo. Como sabemos que $M$ y $N$ son puntos fijos, sus coordenadas no homogéneas deben cumplir la ecuación~\eqref{C6_eq_homografia_segundo_orden_punto_fijo}. Por tanto, una forma de hallar estos parámetros sería sustituir las coordenadas no homogéneas de los puntos fijos en dicha ecuación y ver para que valores de $\alpha, \beta,\gamma,\delta$ se cumple. 

Para facilitar este cálculo tomamos la referencia $\mf{R}=\{M,N;e\}$, donde $e$ es un punto cualquiera distinto de $M$ y $N$. En esta referencia las coordenadas no homogénea de $M$ y $N$ son $\infty$ y $0$, respectivamente. Por tanto, la ecuación~\eqref{C6_eq_homografia_segundo_orden_punto_fijo} admite como soluciones $0$ y $\infty$. Sustituyendo estos valores en la ecuación obtenemos que 
\begin{equation*}
	\alpha\cdot 0+(\beta+\gamma)0+\delta=0\sii\delta=0
\end{equation*}
Hagamos el cálculo para $\infty$ con más cuidado. Sustituyendo las coordenadas no homogéneas por las coordenadas homogéneas en la ecuación~\eqref{C6_eq_homografia_segundo_orden}, esta queda
\begin{equation}
	\alpha xx'+\beta xy'+\gamma x'y+\delta yy'=0
\end{equation}
Dado que estamos con puntos fijos imponemos que $(x',y')=(x,y)$ y que $\delta=0$, pues lo hemos hallado antes
\begin{equation*}
	\alpha x^2+\beta xy+\gamma xy=0
\end{equation*}
Las coordenadas homogéneas de $M$ respecto a la referencia $\mf{R}$ son $(1:0)$. Sustituyéndolas en la ecuación anterior obtenemos que
\begin{equation*}
\alpha 1+\beta 0+\gamma 0=0\sii \alpha=0
\end{equation*}
Por tanto, en una homografía de $\proy^1$ con dos puntos fijos se tiene que $\alpha=c=0$ y $\delta=-b=0$. Entonces, la \ti{ecuación canónica} de la homografía respecto a la referencia $\mf{R}$ tomada es
\begin{equation}
	\theta'=\frac{a\theta+b}{c\theta +d}=\frac{a}{d}\theta=k\theta
\end{equation}
donde la constante $k$ se llama \tb{módulo de la homografía}. Notemos que dada cualquier homografía que tenga dos puntos fijos distintos, tomando la referencia adecuada, podemos decir que dichos puntos fijos son el cero y el infinito.

\begin{obs}
	El cálculo para el punto $M$, cuya coordenada no homogénea era $\infty$, puede hacerse sin tener tanto cuidado, gracias a que hemos comprobado rigurosamente que funciona.
	\begin{equation*}
		\alpha\theta^2+(\beta+\gamma)\theta+\delta=0\ra\alpha+(\beta+\gamma)\frac{1}{\theta}+\delta\frac{1}{\theta^2}=0
	\end{equation*}
	\begin{equation*}
		\alpha+(\beta+\gamma)\frac{1}{\infty}+\delta\frac{1}{\infty^2}=0\sii \alpha=0
	\end{equation*}
\end{obs}
Es importante observar que, aunque la ecuación $\theta'=k\theta$ se denomina canónica, en realidad no lo es mucho, ya que esta expresión se ha obtenido tomando la referencia $\mf{R}=\{M,N;e\}$. Notemos que si en vez de tomar como referencia $\mf{R}$ tomamos $\overline{\mf{R}}=\{N,M;e\}$, la ecuación cambia.

En efecto, el cambio de $\mf{R}$ a $\overline{\mf{R}}$ viene dado por la homografía que trasforma $(1:0)$ en $(0:1)$ y $(0:1)$ en $(1:0)$. Sustituyendo estas trasformaciones, en coordenadas no homogéneas, en la ecuación~\eqref{C6_transformada_moebius}, obtenemos que la ecuación de la homografía que transforma $\mf{R}$ en $\overline{\mf{R}}$ es
\begin{equation*}
	\overline{\theta}=\frac{b}{c\theta}=\frac{\lambda}{\theta}
\end{equation*}
Por tanto, la ecuación $\theta'=k\theta$ respecto a la referencia $\overline{\mf{R}}$, que se obtiene despejando en la ecuación anterior $\theta$ en función de $\overline{\theta}$ y sustituyendo, es
\begin{equation*}
	\frac{\lambda}{\overline{\theta}'}=k\frac{\lambda}{\overline{\theta}}\sii \overline{\theta}'=\frac{1}{k}\overline{\theta}
\end{equation*}
por lo que en este caso el módulo de la homografía con puntos fijos $M$ y $N$ sería $1/k$. 

Podríamos preguntarnos entonces porque darle el apellido canónico. Aunque una permutación de los puntos $M$ y $N$ en la referencia nos da un valor del módulo de la homografía distinto, el parámetro $k$ está geométricamente asociado a la homografía, cosa que veremos no ocurrirá con parámetros posteriores. Es decir, solo está ligado a la homografía y al orden de los puntos $M$ y $N$ en la referencia, y dependiendo de este obtendremos $k$ o $1/k$. Observemos que no depende del punto unidad $e$ que hayamos tomado. Por tanto, la ecuación
\begin{equation}
	\label{C6_theta_ktheta}
	\theta'=k\theta
\end{equation}
puede considerarse canónica.

El parámetro $k$ puede ser hallado a través de la razón doble, ya que estamos tratando homografías. En efecto, dada una homografía $h$ que deja fijos los puntos $M$ y $N$ y, por tanto, en la referencia $\mf{R}=\{M,N;e\}$ viene descrita por la ecuación~\eqref{C6_theta_ktheta} para cierto $k$, se tiene que
\begin{equation}
	\{M,N,P,h(P)\}=\{\infty,0,\theta,\theta'=k\theta\}=k
\end{equation}
para un punto $P$ cualquiera independientemente de $e$. Si permutamos los puntos $M$ y $N$, por la simetría de la razón doble se tiene que
\begin{equation}
	\{N,M,P,h(P)\}=\frac{1}{k}
\end{equation}
independientemente de $e$. Esto nos demuestra de nuevo que el módulo depende solo de la homografía y del orden de los puntos fijos.

Hagamos un pequeño inciso que retomaremos más adelante. Si escribimos, en la referencia $\mf{R}=\{M,N,e\}$, la matriz asociada a una homografía $h$ que tiene dos puntos fijos $M$ y $N$ obtenemos
\begin{equation*}
	\left( \begin{array}{cc}
	a&0\\
	0&d
	\end{array}\right) 
	\sim \left( \begin{array}{cc}
	\frac{a}{d}&0\\
	0&1
	\end{array}\right) 
	\sim \left( \begin{array}{cc}
	k&0\\
	0&1
	\end{array}\right)
\end{equation*}
Por tanto, hemos encontrado una referencia en la cual la homografía $h$ es diagonal. Notemos que, además, el módulo de la homografía es el cociente de los autovalores.

\subsection{Un punto fijo doble}
Veamos ahora qué ecuación describe a una homografía $h$ de $\proy^1$ que deja fijo un solo punto. Si $M$ es el punto fijo de $h$, entonces la ecuación~\eqref{C6_eq_homografia_segundo_orden_punto_fijo} debe admitir como solución la coordenada no homogénea de $M$. Por tanto, procederemos como en el caso anterior. Tomaremos una referencia adecuada y sustituiremos la coordenada no homogénea de $M$, respecto a dicha referencia, en la ecuación~\eqref{C6_eq_homografia_segundo_orden_punto_fijo} para hallar los parámetros $\alpha, \beta,\gamma,\delta$.

Dado que solo hay un punto fijo, tomamos como referencia $\mf{R}=\{M,X;e\}$, donde $X$ y $e$ son puntos diferentes y distintos de $M$. En esta referencia la coordenada no homogénea de $M$ es $\infty$. Por tanto, para cualquier homografía con un punto fijo doble, tomando la referencia adecuada, se puede decir que dicha homografía deja fijo el infinito. Dado que ya justificamos anteriormente los cálculo con $\infty$ esta vez lo haremos sin tener cuidado. Recordemos que podíamos escribir
\begin{equation*}
	\alpha\theta^2+(\beta+\gamma)\theta+\delta=0\ra\alpha+(\beta+\gamma)\frac{1}{\theta}+\delta\frac{1}{\theta^2}=0
\end{equation*}
\begin{equation*}
	\alpha+(\beta+\gamma)\frac{1}{\infty}+\delta\frac{1}{\infty^2}=0\sii \alpha=0
\end{equation*}
Por tanto la ecuación que describe los puntos fijos de la homografía pasa a ser
\begin{equation*}
	(\beta+\gamma)\theta+\delta=0
\end{equation*}
Dado que el punto $M$ es solución doble de la ecuación de segundo grado~\eqref{C6_eq_homografia_segundo_orden_punto_fijo}, si volvemos a sustituir su coordenada no homogénea en la ecuación anterior, esta debe cumplirse. Por tanto
\begin{equation*}
	(\beta+\gamma)\theta+\delta=0\ra(\beta+\gamma)+\delta\frac{1}{\theta}=0
\end{equation*}
\begin{equation*}
	(\beta+\gamma)+\delta\frac{1}{\infty}=0\sii \beta=-\gamma
\end{equation*}
Por tanto, en una homografía de $\proy^1$ con un punto fijo doble se tiene que $\alpha=c=0$ y $-a=\beta=-\gamma=-d$. Entonces, la ecuación ``canónica" de la homografía, respecto a la referencia $\mf{R}$ tomada, es
\begin{equation}
\theta'=\frac{a\theta+b}{c\theta +d}=\frac{a\theta+b}{a}=\theta+\mu
\end{equation}
A este tipo de homografías de las denomina \ti{elaciones} y se dice que dejan fijo el infinito pues, como hemos visto antes, tomando la referencia adecuada el punto fijo $M$ pasa a ser $\infty$.

Notemos que esta vez el apellido canónico no es merecido, al contrario que en el caso anterior, pues el parámetro $\mu$ no está geométricamente asociado a la homografía. 

Veámoslo. Partiendo de la referencia $\mf{R}=\{M,X;e\}$, realizamos un cambio de coordenadas que lleve el infinito al infinito. Por ejemplo, aplicamos una homografía con ecuación
\begin{equation*}
	\overline{\theta}=\frac{1}{k}\theta
\end{equation*}
es decir, una homografía $h$ que en esa referencia deje fijos el cero y el infinito. Por tanto, estamos pasando de la referencia $\mf{R}$ a la referencia $\overline{\mf{R}}=\{M,X;h(e)\}$. 

La ecuación $\theta'=\theta+\mu$ respecto a la referencia $\overline{\mf{R}}$, que se obtiene despejando en la ecuación anterior $\theta$ en función de $\overline{\theta}$ y sustituyendo, es
\begin{equation*}
	k\overline{\theta'}=k\overline{\theta}+\mu\sii \overline{\theta'}=\overline{\theta}+\frac{\mu}{k}
\end{equation*}
Si $k=\mu$ entonces la ecuación de una homografía cuyo punto fijo es $M$ respecto a la referencia $\overline{\mf{R}}$ es
\begin{equation*}
	\overline{\theta}'=\overline{\theta}+1
\end{equation*}
Como se puede observar, el parámetro $\mu$ no está asociado a la homografía, pues lo hemos eliminado de la ecuación que la define. Por tanto, la ecuación $\theta'=\theta+\mu$ no se puede considerar canónica. Lo máximo a lo que podemos aspirar es a expresar la ecuación de una elación como $\theta'=\theta+1$, tomando la referencia adecuada.

De nuevo, hagamos un inciso que retomaremos más adelante. Dada una homografía $h$ que tiene un punto fijo doble $M$, si escribimos, en la referencia $\mf{R}$ en la cual la ecuación que la define es $\theta'=\theta+1$ (y por tanto $b=a$), su matriz asociada obtenemos
\begin{equation*}
\left( \begin{array}{cc}
a&a\\
0&a
\end{array}\right) 
\sim \left( \begin{array}{cc}
1&1\\
0&1
\end{array}\right) 
\end{equation*}
Por tanto, hemos encontrado una referencia en la cual la matriz de la homografía $h$ se encuentra en su forma canónica de Jordan, salvo múltiplo.

\section{Formas canónicas de Jordan y puntos fijos}
Hemos anticipado que existe cierta relación entre los puntos fijos de una homografía y la forma canónica de Jordan de su matriz asociada. En este apartado explicaremos cuál es esa relación.\\

Sea $h$ una homografía distinta de la identidad. Si un punto $[v]$ es punto fijo de la homografía, entonces 
\begin{equation}
	\label{C6_eq_puntofijo_autovector}
	[\widehat{h}(v)]=h([v])=[v]\sii \widehat{h}(v)=\rho v
\end{equation}
para cierto $\rho$, de lo cual se deduce que $v$ es autovector de la aplicación lineal $\widehat{h}$ con autovalor asociado $\rho$. Por tanto, un punto $x$ es punto fijo de la homografía si y solo si es un autovector de la aplicación lineal $\widehat{h}$, asociado a un autovalor no nulo.

Antes de continuar hagamos un pequeño recordatorio.

\begin{obs}
	Dada un endomorfismo $\widehat{h}$ de $E$ ($\dim(E)=n$) y sean $\lambda_i$, con $i=1,\cdots,r$ sus distintos autovalores, se llama subespacio propio asociado a $\lambda_i$ al subesapcio vectorial $V_{\lambda_i}$ formado por todos los autovectores asociados a dicho autovalor:
	\begin{equation}
		V_{\lambda_i}:=\{u\in E\tq \widehat{h}(u)=\lambda_i u\}
	\end{equation}
	Se verifica que 
	\begin{equation}
		V_{\lambda_i}= \ker(\widehat{h}-\lambda_i I) \ ; \quad \dim(V_{\lambda_i})=n-rg(A-\lambda_i I)
	\end{equation}
	donde $A$ es la matriz asociada a $\widehat{h}$. 
	
	La dimensión de $V_{\lambda_i}$, denotada por $d_i$ se denomina multiplicidad geométrica. Por otro lado, la multiplicidad de $\lambda_i$ como raíz del polinomio característico se denomina multiplicidad algebraica, y se denota por $\alpha_i$. Si $\alpha_i+\cdots+\alpha_r=n$ y $d_i=\alpha_i$ para cada $i=1,\cdots,r$, entonces $\widehat{h}$ es diagonalizable. En el caso en el que solo se cumpla que $\alpha_i+\cdots+\alpha_r=n$ el endomorfismo será ``Jordanizable".
\end{obs}

Sabemos pues que $u$ es autovector de $\widehat{h}$ asociado a un autovalor $\lambda$ no nulo si y solo si $[u]$ es punto fijo. Por tanto, como $V_{\lambda}$ está formado por todos los autovectores asociados a $\lambda$, se tiene que $\proy(V_\lambda)$ es una variedad proyectiva formada por punto fijos de $h$. Además, el conjunto de las proyecciones de los subespacios propios asociados a los distintos autovalores de $\widehat{h}$ nos proporcionan todos los puntos fijos de $h$.

Notemos que hasta ahora no hemos utilizado en absoluto que $h$ sea una homografía de $\proy^1$. Con esto queremos resaltar que todo lo dicho es válido para homografías de $\proy^n$. Es decir, los puntos fijos de una homografía de $\proy^n$ vienen dados por $\proy(V_{\lambda_i})$ para los distintos autovalores $\lambda_i$ de la aplicación lineal asociada.

\begin{obs}
	Cualquier homografía de un espacio proyectivo complejo tiene puntos fijos. Sin embargo, hay homografías del espacio proyectivo real que no los tienen. Podemos precisar un poco más, ya que sabemos que todo polinomio con coeficientes en $\R$ de grado impar tiene raíces. Por tanto para que haya homografías sin puntos fijos, el grado del polinomio debe ser par, con lo cual la dimensión del espacio vectorial es par y, por tanto, la dimensión del espacio proyectivo debe ser impar. A no ser que se diga lo contrario, de aquí en adelante trabajaremos en $\C$.
\end{obs}

Indaguemos un poco más para obtener las conclusiones del apartado anterior, es decir, que una homografía $h$ de $\proy^1$ distinta de la identidad tiene o dos puntos fijos distintos o uno doble. Para ello trabajemos con matrices.\\

Sea $A$ la matriz asociada a la homografía $h$, que no es más que la matriz asociada a la aplicación lineal $\widehat{h}$, y $x$ un punto fijo de $h$. Entonces $\rho x=Ax$, con lo cual se obtiene el sistema $(A-\rho I)x=0$, que tiene solución no idénticamente nula si y solo si $\det(A-\rho I)=0$. Este determinante no es otro que el polinomio característico, cuyas raíces, $\rho$, son los autovalores de la aplicación lineal $\widehat{h}$. Dado que nos encontramos en $\proy^1$, la matriz $A$ será una matriz $2\times 2$, y, por tanto, el polinomio característico será un polinomio de segundo grado. Esto implica que tendrá dos soluciones distintas o una solución doble, siempre que el cuerpo sobre el que estemos trabajando se $\C$. Consecuentemente, la aplicación lineal $\widehat{h}$ tendrá un autovalor o dos distintos. Evaluemos cada caso:
\begin{enumerate}
	\item Si $\widehat{h}$ tiene dos autovalores distintos, entonces $\dim(V_{\rho_i})=1$. Por tanto, las variedades proyectivas $\proy(V_{\rho_1})$ y $\proy(V_{\rho_2})$ tienen dimensión cero, por lo que son puntos. Esto da lugar a dos puntos fijos distintos en la homografía. Además si escribimos la matriz $A$ en la base formada por estos dos autovectores obtendremos una matriz diagonal, como habíamos visto antes.
	
	\item Si $\widehat{h}$ tiene un solo autovalor $\rho$, en cuyo caso $\alpha_1=2$, en principio pueden darse dos casos.
	\begin{itemize}
		\item El caso en el que $\dim(V_{\rho})=2=d_1$ no es admisible, ya que hemos supuesto que $h$ es distinta de la identidad. Sin embargo, la aplicación $\widehat{h}$ con $\dim(V_{\rho})=2$ tiene como autovectores asociados a $\rho$ todo el espacio vectorial, por lo que todos los puntos de $\proy^1$ serían puntos fijos de $h$.
		
		\item Si $\dim(V_{\rho})=1=d_1$ entonces $\proy(V_{\rho})$ es un punto. En tal caso, la homografía $h$ tendrá un único punto fijo doble. Además, dado que $\alpha_1=2\not=d_1=1$, la matriz $A$ no es diagonalizable. Sin embargo, sí admite forma canónica de Jordan. Por tanto, en cierta base, a la cual pertenece dicho autovector, la matriz $A$ se encuentra en su forma canónica de Jordan, como vimos anteriormente.
	\end{itemize}
\end{enumerate}
Con todo ello podemos concluir que una homografía de $\proy^1$, distinta de la identidad, tiene dos puntos fijos distintos o un punto fijo doble. Si escribimos la matriz asociada a la homografía en la base dada por dichos puntos fijos obtendremos, en el primer caso, una matriz diagonal y, en el segundo, su forma canónica de Jordan. 
\begin{obs}
	Sea una homografía $h$ de $\proy^1$ y su aplicación lineal asociada $\widehat{h}$. Un punto $[v]$ de $\proy^1$ pertenece al centro de $h$ si y solo si es un autovector de $\widehat{h}$ con autovalor $0$.
	
	En efecto, que $[v]$ esté en el centro de $h$ equivale a que $v\in \ker\widehat{h}$, es decir, que $\widehat{h}(v)=0=0\cdot v$.
\end{obs}

\section{Aplicaciones afines}
En esta sección y en la que sigue trataremos algunos casos particulares, de gran importancia, de homografías de $\proy^1$.

Estudiemos las homografía de $\proy^1$ tales que su matriz asociada es
\begin{equation*}
	A=\left( \begin{array}{cc}
		a&b\\
		0&d
	\end{array}\right)
\end{equation*}
es decir, $c=0$, y con ello , como $d$ no puede ser cero, 
\begin{equation}
	\theta'=\frac{a\theta+b}{d}=\alpha\theta+\beta
\end{equation}
donde $\alpha=a/d$ y $\beta=b/d$. Veamos cuáles son sus puntos fijos.\\

Siguiendo lo visto en el apartado anterior, debemos encontrar los autovalores de $A$ y sus autovectores asociados. Realizando unas sencillas cuentas se llega a que los autovalores son $a$ y $d$. Dado que hay dos autovalores distintos, la homografía $h$ tendrá dos puntos fijos. 

Calculemos dos autovectores, $u_a$ y $u_d$ asociados a estos autovalores. Recordemos que, como $V_{\lambda}= \ker(\widehat{h}-\lambda I)$, si $u$ es un autovector con autovalor $\lambda$, entonces $u\in V_{\lambda}= \ker(\widehat{h}-\lambda I)$, y, por tanto, $(A-\lambda I)u=0$. Esto nos proporciona una forma de hallar los autovectores. En nuestro caso
\begin{equation*}
	(A-a I)\left( \begin{array}{c}
	x\\
	y
	\end{array}\right)=\left( \begin{array}{cc}
	0&b\\
	0&d-a
	\end{array}\right)
	\left( \begin{array}{c}
	x\\
	y
	\end{array}\right)=\left( \begin{array}{c}
	0\\
	0
	\end{array}\right)\sii y=0\ra u_{a}=(1,0)
\end{equation*}
\begin{equation*}
	(A-d I)\left( \begin{array}{c}
	x\\
	y
	\end{array}\right)=\left( \begin{array}{cc}
	a-d&b\\
	0&0
	\end{array}\right)
	\left( \begin{array}{c}
	x\\
	y
	\end{array}\right)=\left( \begin{array}{c}
	0\\
	0
\end{array}\right)\sii y=\frac{a-d}{b}x\ra u_b=(b,d-a)
\end{equation*}
Por tanto, los dos puntos fijos de la homografía $h$ son $(1:0)$ y $(b/d:1-a/d)=(\beta:1-\alpha)$, cuyas coordenadas no homogéneas son $\infty$ y $\frac{\beta}{1-\alpha}$. 

Dado que la ecuación $\theta'=\alpha\theta+\beta$ define una aplicación afín de la recta podemos concluir que \tb{las afinidades}, homografías con $c=0$, \tb{dejan fijo el infinito}. 
\begin{obs}
	No se deben confundir estas con las elaciones. Mientras las elaciones tienen un punto fijo doble, que se puede identificar con el infinito en cierta referencia, las homografías con $c=0$ dejan fijo el infinito sea cual sea la referencia en la que nos encontramos. Es decir, si cambiamos de referencia seguirán teniendo como punto fijo el punto $(1:0)$ de dicha referencia.
\end{obs}
Veamos el \tb{recíproco}, si una homografía de $\proy^1$ deja fijo el infinito, entonces es una aplicación afín. 

Sea una homografía de $\proy^1$ con matriz asociada
\begin{equation*}
	A=\left( \begin{array}{cc}
		a&b\\
		c&d
	\end{array}\right)
\end{equation*}
Si deja fijo el infinito, entonces
\begin{equation*}
	\left( \begin{array}{cc}
	a&b\\
	c&d
	\end{array}\right)
	\left( \begin{array}{c}
	1\\
	0
	\end{array}\right)=\rho\left( \begin{array}{c}
	1\\
	0
	\end{array}\right)\sii a=\rho \ , \ c=0
\end{equation*}
Por tanto, la matriz de la homografía, dado que $d\not=0$, es
\begin{equation*}
	A=\left( \begin{array}{cc}
		a&b\\
		0&d
	\end{array}\right)\sim 
	\left( \begin{array}{cc}
		\alpha&\beta\\
		0&1
	\end{array}\right)
\end{equation*}
Si describimos la homografía que deja fijo el infinito a través de la transformación de Möbius obtenemos que 
\begin{equation*}
	\theta'=\alpha\theta+\beta
\end{equation*}
es decir, es una transformación afín de la recta.\\

Estos dos resultados dan lugar al siguiente lema.
\begin{lem}\label{C6_lem_infinito_p1_sii_afinidad}
	Dada una homografía de $\proy^1$ y $\mf{R}=\{x_0,x_1;e\}$ una referencia arbitraria. Entonces $h$ deja fijo el infinito, es decir, deja fijo el punto $x_0$, si y solo si $c=0$, o equivalentemente, si y solo si es una afinidad.
\end{lem}

El cálculo de los puntos fijos de la homografía de $\proy^1$ dada al principio de esta sección, aquella cuya matriz asociada es
\begin{equation*}
	A=\left( \begin{array}{cc}
		a&b\\
		0&d
	\end{array}\right)
\end{equation*}
y que viene descrita a través de la ecuación
\begin{equation*}
	\theta'=\alpha\theta+\beta
\end{equation*}
podría haberse hecho como se hizo en el primer apartado de este capítulo, cuando aún no conocíamos la relación entre los puntos fijos y los autovectores.

En tal caso, tendríamos que buscar las coordenadas no homogéneas $\theta$ que cumplen la ecuación
\begin{equation*}
	\theta=\alpha\theta+\beta
\end{equation*}
Sin embargo, esta ecuación, dado que es de primer orden, solo tiene una solución, pero sabemos que el $\infty$, es decir el $(1:0)$, es también punto fijo de $h$. Por tanto, debe cumplir también la ecuación. Si hacemos la comprobación con las coordenadas no homogéneas obtenemos que 
\begin{equation*}
	\infty=\alpha\infty+\beta=\infty
\end{equation*}
Sin embargo, personalmente, esta comprobación no me tranquiliza, dado que es un procedimiento falto de rigor. Por tanto, vamos a comprobarlo de nuevo con las coordenadas homogéneas. En tal caso tenemos la ecuación
\begin{equation}
	\label{C6_puntofijo_malcalculo}
	\frac{x'}{y'}=\alpha\frac{x}{y}+\beta
\end{equation}
\begin{equation}
	\label{C6_puntofijo_buencalculo}
	x'y=\alpha xy'+\beta yy'
\end{equation}
Imponiendo que $(x,y)$ es punto fijo obtenemos 
\begin{equation*}
	 xy=\alpha xy+\beta y^2\sii xy-\alpha xy-\beta y^2=0
\end{equation*}
Una simple comprobación nos permite asegurar que el punto $(1:0)$ es punto fijo de $h$
\begin{equation*}
	1\cdot 0-\alpha\cdot 1\cdot 0+\beta\cdot 0=0
\end{equation*}
\begin{obs}
	Si en vez imponer la condición de punto fijo, $(x,y)=(x',y')$, en la ecuación~\eqref{C6_puntofijo_buencalculo} lo hacemos en la ecuación~\eqref{C6_puntofijo_malcalculo} y luego multiplicamos por $y$, el punto $(1:0)$ solo es solución de la ecuación resultante si $\alpha=1$, cuando debería serlo para todo $\alpha$.
	\begin{equation*}
		\frac{x}{y}=\alpha\frac{x}{y}-\beta\sii x=\alpha x+\beta y
	\end{equation*}
	\begin{equation*}
		 1=\alpha 1+\beta 0=\alpha
	\end{equation*}
	Esto nos muestra que trabajar con las coordenadas no homogéneas y pasar después a homogéneas tiene sus limitaciones, y siempre que hagamos cálculos de este estilo debemos tener bien claro qué estamos haciendo.
\end{obs}

\subsection{Aplicaciones afines y homografías de $\proy^n$}
El lema~\ref{C6_lem_infinito_p1_sii_afinidad} se puede generalizar a una homografía de $\proy^n$. A continuación expondremos y demostraremos dicho resultado.

\begin{prop}
	Sea $h$ una homografía de $\proy^n$ y $\mf{R}=\{x_0,\cdots,x_n;e\}$ una referencia arbitraria. El hiperplano $H$ dado por la ecuación implícita $x_n=0$ se denomina plano del infinito. Si $h$ deja invariante el plano del infinito entonces es una aplicación afín.
\end{prop}
\begin{obs}
	Que una homografía deje invariante un hiperplano $H$ no significa que deje invariante cada punto del hiperplano, es decir, que los puntos de $H$ sean puntos fijos, sino que la imagen de un punto de $H$ pertenece al hiperplano, pero no tiene por qué ser el mismo punto.
\end{obs}
\begin{proof}
	Sea $h:\proy^n\rightarrow \proy^n$ una homografía que deja invariante el plano $H$ dado por la ecuación $x_n=0$. La matriz asociada a dicha aplicación vendrá dada por
	\begin{equation*}
		\left( \begin{array}{ccc}
			a_{00}&\cdots&a_{0n}\\
			a_{10}&\cdots&a_{1n}\\
			\vdots&\ddots&\vdots\\
			a_{n0}&\cdots&a_{nn}
		\end{array}\right) 
	\end{equation*}
	para ciertos $a_{ij}$, con $i,j=0,\cdots,n$. Dado un punto $(x_0:\cdots:x_n)$ su imagen vendrá dada por la clase del vector $(x_0',\cdots,x_n')$ tal que
	\begin{equation*}
		\left( \begin{array}{ccc}
			a_{00}&\cdots&a_{0n}\\
			a_{10}&\cdots&a_{1n}\\
			\vdots&\ddots&\vdots\\
			a_{n0}&\cdots&a_{nn}
		\end{array}\right) 
		\left( \begin{array}{c}
			x_0\\
			x_1\\
			\vdots\\
			x_n
		\end{array}\right) =\rho
		\left( \begin{array}{c}
			x_0'\\
			x_1'\\
			\vdots\\
			x_n'
		\end{array}\right)
	\end{equation*}
	Los vectores pertenecientes a $\widehat{H}$ tienen la forma $(x_0,\cdots,x_{n-1},0)$. Dado que $h$ deja invariante el hiperplano $H$ debe cumplirse que
	\begin{equation*}
		\left( \begin{array}{ccc}
			a_{00}&\cdots&a_{0n}\\
			a_{10}&\cdots&a_{1n}\\
			\vdots&\ddots&\vdots\\
			a_{n0}&\cdots&a_{nn}
		\end{array}\right) 
		\left( \begin{array}{c}
			x_0\\
			\vdots\\
			x_{n-1}\\
			0
		\end{array}\right) =\rho
		\left( \begin{array}{c}
			x_0'\\
			\vdots\\
			x_{n-1}'\\
			0
		\end{array}\right)
	\end{equation*}
	Por tanto, debe cumplirse la ecuación
	\begin{equation*}
		a_{n0}x_0+\cdots+a_{nn-1}x_{n-1}+a_{nn}\cdot0=\rho\cdot0=0
	\end{equation*}
	para todo $(x_0,\cdots,x_{n-1})\not=(0,\cdots,0)$. Esto implica que, necesariamente
	\begin{equation*}
		a_{n0}=0=a_{n1}=\cdots=a_{nn-1}
	\end{equation*}
	con lo que la matriz asociada a $h$ adopta la forma
	\begin{equation*}
		\left( \begin{array}{cccc}
			a_{00}&\cdots&a_{0n-1}&a_{0n}\\
			a_{10}&\cdots&a_{1n-1}&a_{1n}\\
			\vdots&\ddots&\vdots&\vdots\\
			0&\cdots&0&a_{nn}
		\end{array}\right) 
	\end{equation*}
	Dado que $a_{nn}\not=0$, pues debe ser invertible, la matriz se puede expresar, usando bloques, como
	\begin{equation*}
		\left( \begin{array}{ccc|c}
			&&&\\
			&A&&b\\
			&&&\\ \hline
			0&\cdots&0&1
		\end{array}\right) =
		\left( \begin{array}{ccc|c}
			&&&\\
			&A&&b\\
			&&&\\ \hline
			&0^t&&1
		\end{array}\right)
	\end{equation*}
	Veamos cuál es la expresión de una homografía de $\proy^n$ con esta matriz asociada. Dado un punto $p=(x_0:\cdots:x_{n-1}:x_n)\not\in H$, es decir, tal que $x_n\not=0$, podemos escribirlo como
	\begin{equation*}
		p=(x_0:\cdots:x_{n-1}:x_n)=\left( \frac{x_0}{x_n}:\cdots:\frac{x_{n-1}}{x_n}:1\right) =(X_0:\cdots:X_{n-1}:1)=
		\left( \begin{array}{c}
			\\
			X\\
			\\
			1
		\end{array}\right)^t
	\end{equation*}
	donde $X_i=\frac{x_i}{x_n}$ para $i=0,\cdots,n-1$. Entonces
	\begin{equation*}
		\left( \begin{array}{ccc|c}
			&&&\\
			&A&&b\\
			&&&\\ \hline
			&0^t&&1
		\end{array}\right)
		\left( \begin{array}{c}
			\\
			X\\
			\\
			1
		\end{array}\right)=
		\left( \begin{array}{c}
			\\
			AX+b\\
			\\
			1
		\end{array}\right)=
		\left( \begin{array}{c}
			\\
			X'\\
			\\
			1
		\end{array}\right)
	\end{equation*}
	Con ello, la homografía $h$ que deja fijo el plano del infinito $H$ viene descrita por
	\begin{equation}
		h:X'=AX+b
	\end{equation}
	siendo por tanto $h$ una aplicación afín.
\end{proof}

\section{Involuciones}
En esta sección introduciremos un tipo de homografías, llamadas involuciones, que serán de gran importancia a la hora de estudiar cónicas.
\begin{defi}
	Dada una homografía $h$ de $\proy^1$, distinta de la identidad, se dice que es una \ti{involución} si $h^2=id$.
\end{defi}
Estudiemos cómo se caracterizan las involuciones y cómo describirlas.
\begin{lem}
	Dada una homografía $h$ de $\proy^1$ distinta de la identidad. Existe un punto $P\in\proy^1$ tal que $h(P)\not=P$ y $h^2(P)=P$, si y solo si $h$ es una involución.
\end{lem}
\begin{proof}
	Sea $h:\proy^1\rightarrow\proy^1$ una homografía, distinta de la identidad. Trivialmente, si $h$ es una involución entonces $h^2=id$, y, por tanto, para cualquier punto $P\in\proy^1$ tal que $h(P)\not=P$ se tiene que $h^2(P)=P$.\\
	
	Hagamos pues la otra implicación.
	
	Supongamos que existe un punto $P\in\proy^1$ que cumple $h(P)\not=P$ y $h^2(P)=P$. Sea $P'=h(P)$, entonces lo anterior es equivalente a que $h(P)=P'$ y $h(P')=P$, con $P'\not=P$.
	
	Tomemos $\mf{R}=\{P,P';e\}$ como referencia de $\proy^1$ y encontremos la expresión de la ecuación general de la cónica en esta referencia. Notemos que en este referencia las coordenadas del punto $P$ son $(1:0)$, mientras que las del punto $P'$ son $(0:1)$. Recordemos que la ecuación general en coordenadas homogéneas se podía escribir como
	\begin{equation*}
		\alpha xx'+\beta xy'+\gamma x'y+\delta yy'=0
	\end{equation*}
	donde debemos determinar los parámetros $\alpha,\beta,\gamma$ y $\delta$. Dado que $h(P)=P'$ el par $(x,y)=(1:0)$ y $(x',y')=(0:1)$ debe satisfacer la ecuación anterior. Sustituyendo obtenemos que $\beta=0$. Dado que $h(P')=P$, lo mismo ocurre con el par $(x,y)=(0:1)$ y $(x',y')=(1:0)$. Sustituyendo se obtiene que $\gamma=0$. Con ello la ecuación general de la homografía queda
	\begin{equation*}
		\alpha xx'+\delta yy'=0\sii \alpha\theta\theta'+\delta=0\sii \theta\theta'=\lambda\sii \theta'=\frac{\lambda}{\theta}
	\end{equation*}
	donde $\lambda=-\delta/\alpha$. 
	
	Dado un punto $P_0\in \proy^1$ arbitrario tal que $h(P_0)\not=P_0$, es decir, que no sea un punto fijo de la homografía. Sean $\theta_0,\theta_0'$ y $ \theta'$ las coordenadas no homogéneas de $P_0,h(P_0)$ y $h^2(P_0)$ respectivamente. Entonces
	\begin{equation*}
		\theta_0'=\frac{\lambda}{\theta_0}\ra \theta'=\frac{\lambda}{\theta_0'}=\theta_0
	\end{equation*}
	concluyendo así que $h^2(P_0)=P_0$. 
	
	Por tanto, dado cualquier punto $Q$ de $\proy^1$, distinto de los puntos fijos de $h$ pues para ellos se cumple trivialmente, se tiene que $h^2(Q)=Q$. Con esto se concluye que $h^2=id$ y, por tanto, $h$ es involución.
\end{proof}
Esta demostración nos proporciona una información adicional: la ecuación general de una involución respecto a una referencia de la forma $\mf{R}=\{P,h(P);e\}$, donde $P$ no es un punto fijo, es
\begin{equation}
 \theta\theta'=\lambda
\end{equation}
Esta ecuación no es canónica pues, como ocurría con el parámetro de las homografías con un punto fijo doble, $\lambda$ no está geométricamente asociado a la involución.

Otra característica importante de las involuciones es la siguiente.
\begin{lem}
	Sea $h$ una homografía de $\proy^1$ distinta de la identidad y sea $A$ su matriz asociada. Entonces $h$ es una involución si y solo si $Tr(A)=0$.
	\begin{equation*}
		A=\left( \begin{array}{cc}
			a&b\\
			c&d
		\end{array}\right)
	\end{equation*}
\end{lem}
\begin{proof}
	\bra \quad  Sea $h:\proy^1\rightarrow\proy^1$ una involución. Por ser una homografía su ecuación general viene dada por  
	\begin{equation*}
		\alpha\theta\theta'+\beta\theta+\gamma\theta'+\delta=0
	\end{equation*}
	ecuación que cumple la coordenada no homogénea $\theta$ que se transforma en $\theta'$.
	
	Dado que es involución $\theta'$ se transforma en $\theta$ con lo que se debe cumplir
	\begin{equation*}
		\alpha\theta'\theta+\beta\theta'+\gamma\theta+\delta=0
	\end{equation*}
	Restando y operando se llega a que 
	\begin{equation*}
		(\beta-\gamma)(\theta-\theta')=0
	\end{equation*}
	Por tanto, para que esto se cumpla para todo $\theta$, se tiene que $\beta=\gamma$, pues $h$ no es la identidad. Dado que $\beta=-a, \ \gamma=d$ esto implica que $d=-a$ y con ello la matriz asociada a la involución es
	\begin{equation*}
		A=\left( \begin{array}{cc}
			a&b\\
			c&-a
		\end{array}\right)
	\end{equation*}
	con traza nula. Se concluye así que si $h$ es una involución entonces $Tr(A)=0$.\\
	
	\bla \quad Sea $h:\proy^1\rightarrow\proy^1$ una homografía cuya matriz asociada tiene traza nula, es decir, $d=-a$. Dado que $\beta=-a, \ \gamma=d$, esto implica que la ecuación general de la homografía es simétrica respecto a $\theta$ y $\theta'$, es decir
	\begin{equation*}
		\alpha\theta\theta'+\beta(\theta+\theta')+\delta=0
	\end{equation*}
	Sea entonces un punto $P$ tal que $h(P)\not=P$ y sean $\theta,\theta'$ las coordenadas no homogéneas de $P$ y $h(P)$ respectivamente, que cumplen la ecuación anterior pues $\theta$ se trasforma en $\theta'$.  Dado que es simétrica respecto a $\theta$ y $\theta'$ podemos intercambiar las coordenadas y la ecuación se seguirá cumpliendo.
	\begin{equation*}
		\alpha\theta'\theta+\beta(\theta'+\theta)+\delta=0
	\end{equation*}
	Pero esta ecuación implica que $\theta'$ se trasforma en $\theta$, es decir que $h(h(P))=P$. 
	
	Dado que $P$ es un punto arbitrario distinto de los puntos fijos de $h$, se concluye que $h^2=id$ y con ello $h$ es una involución.
\end{proof}

A lo largo de estas dos demostraciones hemos estado excluyendo los puntos fijos de $h$. Pero ¿cuántos son esos puntos fijos? En principio, dado que una involución es una homografía, si nos encontramos en $\C$, podrá tener dos puntos fijos o un punto fijo doble. Sin embargo, como veremos a continuación, las involuciones no pueden tener un punto fijo doble.
\begin{lem}
	\label{C6_lem_involucion_puntosfijos}
	Si una involución tiene puntos fijos, entonces han de ser dos puntos distintos.
\end{lem}
\begin{proof}
	Recordemos que si estamos en $\R$ una homografía puede no tener puntos fijos. Supongamos que la involución $h:\proy^1\rightarrow\proy^1$ sí tiene puntos fijos, cosa que siempre ocurre si $\K=\C$. Recordemos que si tomamos como referencia la formada por los puntos fijos, la matriz $A$ asociada a la involución es diagonal, en el caso de tener dos puntos fijos, o de la forma
	\begin{equation*}
		\left( \begin{array}{cc}
			a&a\\
			0&a
		\end{array}\right) 
	\end{equation*}
	para cierto $a$, en el caso de tener un punto fijo doble. Dado que $h$ es una involución, su matriz debe ser de traza nula, y esto solo ocurre si es diagonal, con autovalores $\lambda$ iguales y de signo opuesto. 
	
	También se puede llegar a la misma conclusión, que la matriz $A$ es necesariamente diagonal y con ello $h$ tiene dos puntos fijos, teniendo en cuenta que como $h^2=id$ entonces $A^2=\rho I$, con $\rho=\lambda^2$
\end{proof}
Por tanto, las involuciones pertenecen al primer grupo de homografías tratadas en este capítulo. Trabajemos en $\C$. Recordemos que para las homografías con dos puntos fijos habíamos definido el módulo de la homografía. Hallemos el valor de $k$ para una involución $h$ a través de la razón doble en el siguiente lema.
\begin{lem}\label{C6:lem_modulo1_sii_involucion}
	Dada una homografía $h$ de $\proy^1$, esta tiene dos puntos fijos con módulo $k=-1$ si y solo si $h$ es una involución.
\end{lem}
\begin{proof}
	\bla \quad Sea $h:\proy^1\rightarrow \proy^1$ una involución. Por el lema~\ref{C6_lem_involucion_puntosfijos} $h$ tiene dos puntos fijos. Basta comprobar que $k=-1$. Para ello, tomemos la referencia $\mf{R}=\{M,N;P\}$ donde $M$ y $N$ son los puntos fijos de $h$ y $P$  es un punto arbitrario distinto de $M$ y $N$. Entonces
	\begin{equation*}
		\{M,N,P,h(P)\}=k
	\end{equation*}
	Como las homografías preservan la razón doble se tiene que 
	\begin{equation*}
		k=\{M,N,P,h(P)\}=\{h(M),h(N),h(P),h(h(P))\}=\{M,N,h(P),P\}=\frac{1}{k}
	\end{equation*}
	Con ello $k=1/k\sii k^2=1\sii k=-1,k=1$. Si $k=1$, entonces eso significaría, por las propiedades de la razón doble, que $P=h(P)$ para todo $P$. Pero $h$ no es la identidad. Por tanto $k=-1$.\\
	
	\bra \quad Sea $h:\proy^1\rightarrow \proy^1$ una homografía con dos puntos fijos $M$ y $N$ tal que $k=-1$, es decir, dado un punto $P$ arbitrario distinto de $M$ y $N$ 
	\begin{equation*}
		\{M,N,P,h(P)\}=-1
	\end{equation*}
	en la referencia $\mf{R}=\{M,N;P\}$. Por la simetría de la razón doble se tiene que
	\begin{equation*}
		\{M,N,h(P),P\}=-1
	\end{equation*}
	Además, dado que las homografías preservan la razón doble
	\begin{equation*}
		\{M,N,P,h(P)\}=\{h(M),h(N),h(P),h(h(P))\}=\{M,N,h(P),h^2(P)\}=-1
	\end{equation*}
	por lo que 
	\begin{equation*}
		\{M,N,h(P),P\}=-1=\{M,N,h(P),h^2(P)\}
	\end{equation*}
	Esto se cumple si $h^2(P)=P$. Dado que $P$ es arbitrario se concluye que $h$ es una involución.
\end{proof}

Concluimos así que $h$ es una involución si y solo si el par $(P,h(P))$ está separado armónicamente del par $(M,N)$ de puntos fijos.

Recordando lo estudiado al principio de este capítulo, y dado que una involución tiene dos puntos fijos y $k=-1$, se tiene que la ecuación canónica, respecto a la referencia $\mf{R}=\{M,N;e\}$, de la involución es
\begin{equation}
	\theta'=-\theta
\end{equation}
Se deduce inmediatamente del lema anterior que si una homografía con dos puntos fijos tiene como ecuación canónica $\theta'=-\theta$, entonces es una involución.\\

Por último, veremos cómo caracterizar las involuciones a partir de dos puntos.\\

Dada una homografía de $\proy^1$ sabemos que esta queda determinada por tres puntos y sus respectivas imágenes, pues con ellos es posible determinar la transformación de Möbius que la caracteriza (o equivalentemente la matriz asociada). Sin embargo, para que una involución quede determinada basta con dos puntos, y sus imágenes, dado que la traza de su matriz es nula y, por tanto, hay un parámetro menos ($d=-a$).

Esto puede comprobarse, dados dos puntos, con coordenadas no homogéneas $\theta_1$ y $\theta_2$, y sus imágenes ($\theta'_1$ y $\theta'_2$), resolviendo las ecuaciones resultantes de sustituir dichos puntos en 
\begin{equation}
	\label{C6:eq_moebius_haces_ec2}
	\theta'=\frac{a\theta+b}{c\theta -a}
\end{equation}
para determinar $a,b$ y $c$, y, con ello, la involución. Sin embargo, este procedimiento es muy tosco. Recordamos que la ecuación general de una involución viene dada por 
\begin{equation*}
	\alpha\theta\theta'+\beta(\theta+\theta')+\delta=0
\end{equation*}
Podríamos sustituir en esta ecuación los puntos y sus imágenes y resolver el sistema lineal resultante, determinando así $\alpha,\beta$ y $\delta$ y con ello la involución:
\begin{equation*}
	\begin{split}
		\alpha\theta_1\theta_1'+\beta(\theta_1+\theta_1')+\delta&=0\\
		\alpha\theta_2\theta_2'+\beta(\theta_2+\theta_2')+\delta&=0
	\end{split}
\end{equation*}
Esto se puede escribir a través de un determinante
\begin{equation}
	\left| \begin{array}{ccc}
		\theta\theta'&\theta+\theta'&1\\
		\theta_1\theta_1'&\theta_1+\theta_1'&1\\
		\theta_2\theta_2'&\theta_2+\theta_2'&1
	\end{array}\right| =0
\end{equation}
El resultado de este determinante no es otro que la ecuación general de una involución que transforma $\theta_1$ en $\theta'_1$ y $\theta_2$ en $\theta'_2$. En efecto, es claro que el resultado es la ecuación general de una involución. Para que esta transforme $\theta_1$ en $\theta'_1$ y $\theta_2$ en $\theta'_2$ basta con que la ecuación general particularizada a esos puntos de cero. Al particularizar lo que estamos haciendo es sustituir la primera fila del determinante por la segunda o la tercera, luego es cero.

Así, para determinar una involución a partir de dos puntos basta con realizar este determinante, en vez de resolver el sistema resultante de la ecuación~\eqref{C6:eq_moebius_haces_ec2}.
\begin{exa}
	Determinar la involución de $\proy^1$ que transforma $\theta_1=1$ en $\theta'_1=-1$ y $\theta_2=0$ en $\theta'_2=2$.\\
	
	Calculamos la ecuación general de dicha involución a través del determinante
	\begin{equation*}
		\left| \begin{array}{ccc}
			\theta\theta'&\theta+\theta'&1\\
			-1&0&1\\
			0&2&1
		\end{array}\right| =0
	\end{equation*}
	Resolviéndolo obtenemos
	\begin{equation*}
		-2\theta\theta'+(\theta+\theta')-2=0
	\end{equation*}
	Esta ecuación general nos permite hallar la transformación de Möbius que caracteriza a la involución, dado que $\alpha=c, \ \beta=-a, \ \delta=-b$ y $d=-a$. Por tanto, la involución buscada queda determinada por
	\begin{equation*}
		\theta'=\frac{-\theta+2}{-2\theta+1}
	\end{equation*}
\end{exa}

\subsection{Haz de ecuaciones de segundo grado}
Debido a que, dado un punto $P\in\proy^1$, si $h$ es una involución de $\proy^1$ se tiene que $h^2(P)=P$, diremos que las involuciones están formadas por pares no ordenados de puntos $(P,h(P)=P')$. Notemos que equivalentemente se tiene $(P',h(P')=P)$.

Una forma de proporcionar pares no ordenados de puntos es mediante una ecuación de segundo grado, ya que nos hallamos en $\C$. Surge entonces la siguiente pregunta: ¿Se podría describir una involución mediante ecuaciones de segundo grado? En esta sección nos dedicaremos al desarrollo de la respuesta.\\

Un par no ordenado de puntos $(P,P')$ viene dado por una ecuación de segundo grado, que podemos escribir como
\begin{equation*}
	a\theta^2+2h\theta+b=0,
\end{equation*}
de tal forma que $\theta$ y $\theta'$, coordenadas no homogéneas de $P$ y $P'$, son las soluciones de la ecuación.

Observemos que si $\theta$ y $\theta'$ son las soluciones de la ecuación anterior, también lo son del doble de ella, o de, en general, $\lambda$ veces ella. Por tanto, un par de puntos no ordenados es equivalente a un rayo de ecuaciones de segundo grado. De esta forma, las ecuaciones de segundo grado pasan a ser puntos y podemos definir un haz.
\begin{defi}[Haz de ecuaciones de segundo grado] Dados dos rayos de ecuaciones de segundo grado
\begin{equation}
	\begin{split}
		S_1&:a_1\theta^2+2h_1\theta+b_1=0\\
		S_2&:a_2\theta^2+2h_2\theta+b_2=0
	\end{split}
\end{equation}
se define el \ti{haz de ecuaciones de segundo grado} asociado a $S_1$ y $S_2$ como el subespacio de dimensión 1 generado por estos rayos
\begin{equation}
	S=S_1+\lambda S_2 \tq \lambda\in\overline{\K}
\end{equation}
\end{defi}
Diremos que un par $(P,P')$ es solución de $S$ si, para cierto $\lambda$, sus coordenadas no homogéneas $\theta$ y $\theta'$ son las soluciones de la ecuación de segundo grado dada por $S_1+\lambda S_2$.

Recordemos que estamos buscando cómo describir las involuciones a partir de ecuaciones de segundo grado. Para ello debemos encontrar una, o unas, ecuaciones de segundo grado cuyas soluciones coincidan con los pares de puntos de la involución.

Dado que las involuciones quedan caracterizadas por dos pares de puntos, lo suyo sería partir de dos pares de puntos, o equivalentemente de dos rayos de ecuaciones de segundo grado. De la misma forma, dado que no son necesarios más de dos pares de puntos para describir una involución, cabe pensar que con dos rayos de ecuaciones de segundo grado podemos conseguir todos los pares de la involución, quedando así descrita. Y estamos en lo cierto.
\begin{prop}
	Sea una involución $h$ de $\proy^1$ y dos de sus pares de puntos no ordenados $(P_1,h(P_1)=P_1')$ y $(P_2,h(P_2)=P_2')$ que se describen a partir de las ecuaciones
	\begin{equation*}
	\begin{split}
		S_1&:a_1\theta^2+2h_1\theta+b_1=0,\\
		S_2&:a_2\theta^2+2h_2\theta+b_2=0.
	\end{split}
	\end{equation*}
	Entonces, la involución queda descrita por el haz de ecuaciones de segundo grado asociado a $S_1$ y $S_2$ 
	\begin{equation*}
		S=S_1+\lambda S_2
	\end{equation*}
\end{prop}
\begin{proof}
	Debemos demostrar que  los pares de la involución $h$ coinciden con las soluciones de $S$. Para ello veamos primero que todos los pares de soluciones de $S$ son pares de $h$, es decir, son de la forma $(P,h(P))$. 
	
	Recordemos que las involuciones tienen dos puntos fijos $M,N$ y que el par $(P,h(P))$ está armónicamente separado del par $(M,N)$ (lema~\ref{C6:lem_modulo1_sii_involucion}). Por tanto, para ver que todas los pares soluciones de $S$ son pares de $h$ basta ver que todos ellos están armónicamente separados del par $(M,N)$, donde $M$ y $N$ son los puntos fijos de $h$.
	
	Sea pues $(\theta,\theta')$ un par solución de $S$ arbitrario, debemos probar que 
	\begin{equation*}
		\{\theta_M,\theta_N,\theta,\theta'\}=-1
	\end{equation*}
	donde $\theta_M,\theta_N$ son las coordenadas no homogéneas de los puntos fijos de $h$. 
	
	Es claro que podemos encontrar un rayo de ecuaciones de segundo grado
	\begin{equation*}
		\overline{S}:\overline{a}\theta^2+2\overline{h}\theta+\overline{b}=0
	\end{equation*}
	cuyas soluciones sean $\theta_M$ y $\theta_N$. Por otro lado, el haz de ecuaciones de segundo grado $S$ viene dado por
	\begin{equation*}
		S:(a_1+\lambda a_2)\theta^2+2(h_1+\lambda h_2)\theta+(b_1+\lambda b_2)=0
	\end{equation*}
	cualquiera que sea $\lambda$.
	
	Por el ejercicio 38 hecho en clase (Ejercicio~\ref{C6:exerc_38}) sabemos que los pares $(\theta,\theta')$ y $(\theta_M,\theta_N)$ están armónicamente separados, que es lo que queremos probar, si y solo si
	\begin{equation*}
		2(h_1+\lambda h_2)\overline{h}-(a_1+\lambda a_2)\overline{b}-(b_1+\lambda b_2)\overline{a}=0
	\end{equation*}
	Dado que esto debe ocurrir para cualquier $\lambda$, pues $(\theta,\theta')$ es arbitrario, están armónicamente separados si y solo si
	\begin{equation*}
		\begin{split}
			2h_1\overline{h}-a_1\overline{b}-b_1\overline{a}=0\\
			2h_2\overline{h}-a_2\overline{b}-b_2\overline{a}=0
		\end{split}
	\end{equation*}
	Por hipótesis, los pares $(\theta_1,\theta'_1)$ y $(\theta_2,\theta'_2)$, coordenadas no homogéneas de los puntos $P_1,P'_1,P_2$ y $P'_2$, son las soluciones de $S_1$ y $S_2$, respectivamente. Por tanto, por el mismo ejercicio, las ecuaciones anteriores se cumplen si y solo si el par $(\theta_M,\theta_N)$ está armónicamente separado de los pares $(\theta_1,\theta'_1)$ y $(\theta_2,\theta'_2)$. Esto último ocurre pues, al ser $(\theta_1,\theta'_1)$, $(\theta_2,\theta'_2)$ las coordenadas no homogéneas de los pares $(P_1,h(P_1)=P_1')$ y $(P_2,h(P_2)=P_2')$ y $h$ involución con puntos fijos $M$ y $N$, se tiene que, como dijimos antes (lema~\ref{C6:lem_modulo1_sii_involucion}),
	\begin{equation*}
		\{\theta_M,\theta_N,\theta_1,\theta_1'\}=-1
	\end{equation*}
	\begin{equation*}
		\{\theta_M,\theta_N,\theta_2,\theta_2'\}=-1
	\end{equation*}
	Por tanto, volviendo hacia atrás se concluye que los pares $(\theta,\theta')$ y $(\theta_M,\theta_N)$ están armónicamente separados, y con ello que los pares soluciones de $S$ son pares de $h$.\\
	
	Falta ver que todos los pares de $h$ son soluciones de $S$. En realidad, esto ya lo hemos demostrado.
	
	En efecto, dado un punto arbitrario $P\in\proy^1$, cuya coordenada no homogénea es $\theta$ y cuya imagen por $h$ tiene coordenada no homogénea $\theta'$, podemos elegir $\lambda$ de tal forma que $\theta$ sea una de las soluciones de $S$. La otra solución de $S$ asociada a dicho $\lambda$ formará con $\theta$ un par $(\theta,\overline{\theta})$ solución de $S$. Por lo que acabamos de demostrar los pares soluciones de $S$ son pares de $h$, y, por tanto, $\overline{\theta}=\theta'$ y con ello el par $(\theta,\theta')$ de $h$ es solución de $S$.
\end{proof}
Por último, veamos el recíproco.
\begin{prop}
	Todo haz de ecuaciones de segundo grado
	\begin{equation*}
		S=S_1+\lambda S_2
	\end{equation*}
	tiene asociada una única involución.
\end{prop}
\begin{proof}
	Sea el haz de ecuaciones de segundo grado $S=S_1+\lambda S_2$ donde
	\begin{equation*}
	\begin{split}
		S_1&:a_1\theta^2+2h_1\theta+b_1=0,\\
		S_2&:a_2\theta^2+2h_2\theta+b_2=0.
	\end{split}
	\end{equation*}
	Sean los pares $(\theta_1,\theta'_1)$ y $(\theta_2,\theta'_2)$ solución de $S_1$ y $S_2$ respectivamente, estos determinan una involución $h$, la cual transforma $\theta_1$ en $\theta'_1$ y $\theta_2$ en $\theta'_2$, quedando así demostrada su existencia.
	
	Además, por la proposición anterior, $h$ queda descrita por el haz $S_1+\lambda S_2$, que es el haz $S$, por lo que todos los pares solución de $S$ son pares de $h$. Por tanto, no existen dos pares que sean solución de $S$ y que estén asociados a otra involución distinta de $h$, quedando demostrada su unicidad.
\end{proof}
Finalmente, y para cerrar este capítulo, veamos un ejemplo para afianzar este nuevo concepto.
\begin{exa}[Haz de ecuaciones de segundo grado] Sean las ecuaciones de segundo grado
	\begin{equation*}
		\begin{split}
			S_1&:(\theta-1)(\theta+1)=0,\\
			S_2&:\theta(\theta-2)=0.
		\end{split}
	\end{equation*}
	¿Cuál es la imagen del punto con coordenada no homogénea $\theta=3$?\\
	
	Sabemos que dadas $S_1$ y $S_2$ hay un única involución $h$ asociada al haz $S_1+\lambda S_2$. Nos piden la imagen de la coordenada no homogénea 3 por $h$, es decir $\theta'=h(3)$. 
	
	La involución $h$ queda determinada por el haz de ecuaciones de segundo grado
	\begin{equation}\label{C6:eq_ejercicio_haz}
		S:(\theta-1)(\theta+1)+\lambda\theta(\theta-2)=0
	\end{equation}
	Todos los pares $(\theta,h(\theta)=\theta')$ son pares de soluciones de $S$. Por tanto, para $\theta=3$ existirá un cierto $\lambda_0$ de tal forma que $(3,h(3))$ sean las soluciones de la ecuación de segundo grado $S_0$ del haz resultante de sustituir $\lambda_0$ en la ecuación~\eqref{C6:eq_ejercicio_haz}. Nuestro objetivo es, por tanto, hallar dicho $\lambda_0$ y después encontrar las soluciones de $S_0$, que serán $3$ y $h(3)$.
	
	Buscamos la ecuación del haz $S$ que tenga como solución $3$. Por tanto se cumple
	\begin{equation*}
		(3-1)(3+1)+3(3-2)\lambda =0\sii \lambda=-\frac{8}{3}
	\end{equation*}
	que es el $\lambda_0$ que buscábamos. La ecuación $S_0$ del haz es
	\begin{equation*}
		S_0:(\theta-1)(\theta+1)-\frac{8}{3}\theta(\theta-2)=0
	\end{equation*}
	\begin{equation*}
		S_0:3(\theta-1)(\theta+1)-8\theta(\theta-2)=0
	\end{equation*}
	cuyas soluciones son $3$ y $1/5$. Por tanto, $h(3)=\theta'=1/5$.
\end{exa}
\begin{exerc}[Ejercicio 38]\label{C6:exerc_38}
	Demuéstrese que si las coordenadas no homogéneas de los pares $(P_1,P_2)$ y $(P_3,P_4)$ están dadas, respectivamente, por las ecuaciones 
	\begin{equation*}
		\begin{split}
			a\theta^2+2h\theta+b=0\\
			a'\theta^2+2h'\theta+b'=0
		\end{split}
	\end{equation*}
	entonces los pares son armónicos si y solo si
	\begin{equation*}
		2hh'-ab'-a'b=0.
	\end{equation*}
\end{exerc}