\chapter{Cónicas y Cuádricas}
\label{C8}
Sea el conjunto de ceros de un polinomio en tres variables conformado únicamente por monomios de grado $2$. Es decir, el conjunto de puntos que satisface la siguiente ecuación.
\begin{equation}
	\label{C8_eq_polinomios1}
	F(x,y,z)=ax^2+by^2+cz^2+2fyz+2gzx+2hxy=0
\end{equation}
Es fácil ver que, si un vector no nulo verifica la ecuación \eqref{C8_eq_polinomios1}, entonces cualquier representante del rayo generado por dicho vector también verifica la ecuación. En efecto, si el punto $u=(x_0,y_0,z_0)$ verifica la ecuación \eqref{C8_eq_polinomios1} se cumple que:
\[F(x_0,y_0,z_0)=ax_0^2+by_0^2+cz_0^2+2fy_0z_0+2gz_0x_0+2hx_0y_0=0\]
Tomemos otro representante de $\class{u}$, es decir, vector de la forma $(\lambda x_0, \lambda y_0, \lambda z_0)$ para cualquier $\lambda$ no nulo. Veamos que, en efecto, también verifica la ecuación:
\[F(\lambda x_0, \lambda y_0, \lambda z_0)=\lambda^2F(x_0,y_0,z_0)=0\]
De esta forma podemos decir que el hecho de que un punto proyectivo esté en el conjunto de ceros de cierto polinomio formado únicamente por monomios de grado $2$, tiene sentido proyectivo. Esto es, si afirmamos que cierto punto proyectivo $\class{u}=(x_0:y_0:z_0)$ pertenece al conjunto de ceros, entonces si cogemos cualquier otro representante, este también pertenecerá a dicho conjunto de ceros.