\section*{Prefacio}
Estas notas son una transcripción (libremente adaptada) de las clases de la asignatura ``\ti{Geometría Lineal}'', impartidas por Antonio Valdés Morales en el curso 2016--2017 a los cursos de tercero de los dobles grados de  Matemáticas -- Física e Ingeniería Informática -- Matemáticas en la facultad de Ciencias Matemáticas de la Universidad Complutense de Madrid (UCM).

Cualquier aportación o sugerencia de mejora es siempre bienvenida.
\subsection*{Requisitos Previos}
Consideramos un requisito indispensable para seguir estas notas tener cierta soltura trabajando con espacios vectoriales de dimensión finita.

No obstante, estas notas incluyen dos apéndices \eqref{AlgebraMatricial} y \eqref{GeometriaVectorial} con los conocimientos que consideramos indispensables, y alguno más que tampoco vendrá mal.

Asimismo se ha procurado elaborar el texto de la forma más autocontenida posible, de modo que un lector un poco más rezagado en ese sentido no debería tener excesivos problemas.
\subsection*{Contenidos Generales}

\subsection*{Contenidos Adicionales}
El espacio proyectivo es un ente que tiene particular interés en la topología. Por ese motivo hemos incluido el apéndice \ref{ProyectivoTopologia}, donde se tratan de desgranar estos asuntos.
\subsection*{Introducción}
\subsection*{Agradecimientos}
Agradecemos las grandes aportaciones de Iván Prada Cazalla a la hora de ilustrar este texto, así como para ayudar a limpiarlo de errores y proponer mejoras significativas.

Asimismo, nos llena de orgullo y satisfacción haber contado con las inteligentes sugerencias de Manuel Navarro García, fruto, sin duda, de una lectura voraz del texto.

También destacables han sido las observaciones y aportaciones de María José Belda Beneyto, Miguel Pascual Domínguez, Javier Pellejero Ortega y Álvaro Rodríguez García.